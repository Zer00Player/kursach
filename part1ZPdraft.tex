\section{Задачи №4 ЕГЭ (преобразование выражений)}

% 4 1)отдельно (формулы) естественно появляется обратная задача. 2)Сослаться на Латех (MathJax) основные команды которыми пользуемся в Латехе texfrag и тп 
% тут рассказать много про MathJax
% закинуть фото ДЛЯ КАЖДОГО пункта
\subsection{Основные используемые команды}

Стандартные команды которыми мы используем из латеха это \texttt{\\cdot} и \texttt{\\mbox}
% потом замени на нормальные описания 

Пример:
\begin{lstlisting}
NAtask.setTask({
			text:
				'В фирме «' + name + '» стоимость (в рублях) колодца из железобетонных колец рассчитывается по формуле $C = ' + plus + '+' + multiply + 
				' \\cdot n$, где $n$ – число колец, ' +
				'установленных при рытье колодца. ' +
				'Пользуясь этой формулой, ' +
				'рассчитайте стоимость колодца из $' + number + '$ колец.',
			answers: cost,
		});
\end{lstlisting}
\textsl{Задача №124.}

\begin{lstlisting}
NAtask.setTask({

			text: 'Количество теплоты (в джоулях), полученное однородным телом при нагревании, ' +
				'вычисляется по формуле $Q = cm(t_2 - t_1)$, где $c$ – удельная теплоёмкость (в Дж),' +
				' $m$ – масса тела (в килограммах), $t_1$ – начальная температура тела (в кельвинах), а $t_2$' +
				' – конечная температура тела (в кельвинах). Пользуясь этой формулой, ' +
				the_orderToFind + ' $Q$ (в джоулях), если $t_2 = ' + t_2 + '$ К, $c = ' + c + '$ $\\frac{\\mbox{Дж}}{\\mbox{кг} \\cdot \\mbox{К}}$,' +
				' $m = ' + m + '$ кг и $t_1 = ' + t_1 + '$ К.',
			answers: Q,

		});
\end{lstlisting}
\textsl{Задача №509609.}

\texttt{\\frac{}{}}, \texttt{\\sqrt{}} и \texttt{\\sin{}} 
самые часто используемые LaTeХ команды и задачах вида 4, для отображения дробей,корней и sin.

Пример:
\begin{lstlisting}
NAtask.setTask({
			text: 'Радиус окружности, описанной около треугольника, ' +
				'можно вычислить по формуле $R = \\frac{a}{2\\sin{\\alpha}}$, где $a$ – сторона, ' +
				'а $\\alpha$ – противолежащий ей угол треугольника. ' +
				'Пользуясь этой формулой, ' + the_orderToFind + ' $' + ['R', 'a'][rand] + '$' +
				', если $' + ['a =' + a, 'R =' + R][rand] + '$ и $\\sin{\\alpha} = \\frac{' + num + '}{' + deNum + '}$.',
			answers: [R, a][rand],
			preference: preference,
		});
\end{lstlisting}
\textsl{Задача №506300.}

\subsection{Адаптирование команд}
Но для более удобной работы с ними в тексте у нас имеется например \texttt{.texfrac()} %и \texttt{.texsqrt}

Пример:
\begin{lstlisting}
NAtask.setTask({

			text: 'Теорему синусов можно записать в виде  $ \\frac{a}{\\sin{\\alpha}} = \\frac{b}{\\sin{\\beta}} $' +
				', где $a$ и $b$ - две стороны треугольника, а $\\alpha$ и $\\beta$ - углы треугольника, лежащие против них соответственно. ' +
				' Пользуясь этой формулой, ' + the_orderToFind + ' ' + '$' + ['\\sin{' + nameSin[0] + '}', nameLetter[0]][rand] + '$' +
				', если ' + '$' + [nameLetter[0] + ' =' + a, '\\sin{' + nameSin[0] + '} =' + sinA.texfrac(1)][rand] + '$' +
				', $' + nameLetter[1] + ' =' + b + '$, $\\sin{' + nameSin[1] + '} = ' + sinB.texfrac(1) + '$.',
			answers: [sinA, a][rand],
			preference: preference,

		});
\end{lstlisting}
\textsl{Задача №530329.}
%Закинуть фото разницы между \\frac{}{} и texfrac() 

(про \texttt{.shuffle()})
Пример:
\begin{lstlisting}
let letter = ['a', 'b', 'c'].shuffle();
\end{lstlisting}
\textsl{Задача №2939.}
\lstinputlisting[]{code/4/2939.js} 

(про \texttt{.sqrt()} \texttt{.cbrt()})

Пример:
\lstinputlisting[]{code/4/512937.js} 


\texttt{isValidTriangle()} - из  flatten-shape-geometry 1.8.2, из 3 переменных проверяет могут ли они составить треугольник. и выдают true - если является, и false -если нет
Пример:
\begin{lstlisting}
	let a = sl(2, 30);
		let b = sl(2, 30);
		let c = sl(2, 30);

		genAssert(isValidTriangle(a, b, c), 'Должно являться треугольником');
\end{lstlisting}
\textsl{Задача №506550.}

\texttt{.rod} - помогает в определении рода существительного (где 0 - мужской,1 - женский, 2 - средний, 3- всегда множественный). Очень помогает когда местоимение зависит от рода существительного.

Пример:
\lstinputlisting[]{code/4/512937.js} 
\textsl{Задача №512937.}

\texttt{.isZ()} проверяет является ли n целым числом.  

Пример:
\begin{lstlisting}
let second = sl(5, 30);
		let amperage = sl(5, 30);
		let voltage = sl(5, 30);
		let resistance = slKrome([amperage, voltage, second], 5, 30);
		let answer = [amperage ** 2 * resistance * second, voltage ** 2 * second / resistance][rand];

		genAssert(answer.isZ(), 'должно быть целым');
\end{lstlisting}
\textsl{Задача №523098.}


%\prototype[separator]{Array}{shuffleJoin}
%Перемешивает и соединяет массив с разделителем \texttt{separator}. \texttt{separator} по умолчанию пустая строка. Функция используется в листинге \ref{lst:3011}
%в строке \ref{line:shuffleJoin} для отображения условий задачи в случайном порядке.
%%\begin{lstlisting}
%    let array = ['A', 'B', 'C', 'D',];
%    array.shuffleJoin();
%    //ADBC

%    array.shuffleJoin('; ');
%    //C; D; B; A 
%\end{lstlisting}

\subsection{Обратные задачи}
про preference % рассказать по больше про обратные задачи из-за preference
\lstinputlisting[]{code/4/530329.js} 

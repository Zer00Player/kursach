\section*{Заключение}
\addcontentsline{toc}{section}{Заключение}
За этот год был полностью покрыт открытый банк заданий ФИПИ по темам:
		      \begin{itemize}
			      \item Планиметрия — 26 шаблонов принято.
			      \item Вектора — 18 шаблонов (10 принято, 8 на рецензировании).
			      \item Стереометрия — 56 шаблонов (7 принято, 49 на рецензировании).
			      \item Теория вероятности — 10 шаблонов на рецензировании.
			      \item Теория вероятности (повышенной сложности) — 11 .шаблонов (1 принят 10 на рецензировании).
		      \end{itemize}

В ядро проекта добавлены: 
\begin{itemize}
    \item Функции, упрощающие написание шаблонов по темам «Планиметрия» и  «Стереометрия».
    \item Класс многогранников.
    \item Линейный проектор из $\mathbb{R}^3 \to \mathbb{R}^2$.
\end{itemize}

А также сокращён технический долг проекта.

Все добавленные в проект задания можно использовать для составления контрольных работ, проведения текущего контроля знаний учащихся, подготовки к ЕГЭ.

В будущем планируется добавить в проект класс плоских геометрических фигур и использовать в заданиях по теме «Планиметрия» динамические изображения.

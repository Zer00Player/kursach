\section*{Заключение}
\addcontentsline{toc}{section}{Заключение}
В ходе выполнеия курсовой работы за 3 курс был покрыт открытый банк заданий ФИПИ по темам:
		      \begin{itemize}
			      \item Текстовые задачи (на смекалку) — 12 шаблонов принято.
			      \item Текстовые задачи (проценты и дроби) — 29 шаблонов принято.
			      \item Преобразования выражений — 29 шаблонов (25 принято 4 на внутреннем рецензировании).
			      \item Задачи с прямыми — 10 шаблонов принято. (3 с рисунком и 7 на сравнение чисел )
		      \end{itemize}

В ядро проекта добавлены: 
\begin{itemize}
    \item Функции, упрощающие написание шаблонов по теме «Координатная прямая».
    \item r2 - род для числительных значений.
\end{itemize}

А также сокращён технический долг проекта.

Все добавленные в проект задания можно использовать для составления контрольных работ, проведения текущего контроля знаний учащихся, подготовки к ЕГЭ.~\cite{chas-ege}

В будущем планируется добавить в проект ещё большее количество заданий и функций для них д.



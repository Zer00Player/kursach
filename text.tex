\documentclass[a4paper, 14pt]{extarticle}

\usepackage{ifluatex}
\usepackage{ifpdf}

\ifluatex
\usepackage{fontspec}
\defaultfontfeatures{Renderer=Basic,Ligatures={TeX}}
\setmainfont{CMU Serif}
\setsansfont{CMU Sans Serif}
\usepackage{polyglossia}
\setdefaultlanguage{russian}
\setotherlanguage{english}
\setmainfont{CMU Serif}
\newfontfamily{\cyrillicfont}{CMU Serif}
\setsansfont{CMU Sans Serif}
\newfontfamily{\cyrillicfontsf}{CMU Sans Serif}
\setmonofont{CMU Typewriter Text}
\newfontfamily{\cyrillicfonttt}{CMU Typewriter Text}
\else
\ifpdf 
\usepackage[T2A]{fontenc}
\usepackage[utf8]{inputenc}
\usepackage[english,russian]{babel}
\typeout{PDF only}
\fi
\fi

\usepackage[left=1cm,right=1cm,top=2cm,bottom=2cm]{geometry}
%%% Дополнительная работа с математикой
\usepackage{amsfonts,amssymb,amsthm,mathtools} % AMS
\usepackage{amsmath}
\usepackage{icomma} % «Умная» запятая: $0,2$ — число, $0, 2$ — перечисление

\usepackage{mathrsfs} % Красивый матшрифт

%% Перенос знаков в формулах (по Львовскому)
\newcommand*{\hm}[1]{#1\nobreak\discretionary{}
  {\hbox{$\mathsurround=0pt #1$}}{}}

%%% Работа с картинками

\usepackage{graphicx}  % Для вставки рисунков
\graphicspath{ {images/} }
\setlength\fboxsep{3pt} % Отступ рамки \fbox{} от рисунка
\setlength\fboxrule{1pt} % Толщина линий рамки \fbox{}
\usepackage{wrapfig} % Обтекание рисунков и таблиц текстом

\usepackage[dvipsnames]{xcolor}
\usepackage{verbatim}
\usepackage{hyperref}

\usepackage{listings}

\lstdefinelanguage{JavaScript}{
  keywords={let, typeof, new, true, false, catch, function, return, null, catch, switch, var, if, in, while, do, else, case, break, const},
  keywordstyle=\color{blue}\bfseries,
  ndkeywords={class, export, boolean, throw, implements, import, this,},
  ndkeywordstyle=\color{darkgray}\bfseries,
  identifierstyle=\color{black},
  sensitive=false,
  comment=[l]{//},
  morecomment=[s]{/*}{*/},
  commentstyle=\color{green}\ttfamily,
  stringstyle=\color{red}\ttfamily,
  morestring=[b]',
  morestring=[b]",
  escapechar=|
}

\lstset{
  language=JavaScript,
  extendedchars=true,
  basicstyle=\small\ttfamily,
  showstringspaces=false,
  breakatwhitespace=true,
  showspaces=false,
  numbers=left,
  numberstyle=\footnotesize,
  numbersep=9pt,
  tabsize=2,
  keepspaces=true,
  breaklines=true,
  showtabs=false,
  captionpos=b
  escapechar=|,
  frame=single ,
  commentstyle=\itshape ,
  frameshape={RYR}{Y}{Y}{RYR},
  stringstyle =\bfseries,
}

\usepackage{autobreak}

\newcommand*{\function}[2]{\texttt{\textcolor{Red}{function} \textcolor{Purple}{#1}(#2)}\linebreak}

\newcommand*{\prototype}[3][]{\texttt{\textcolor{Orange}{#2}.\textcolor{Blue}{prototype}.\textcolor{Purple}{#3} = \textcolor{Red}{function}(#1)}\linebreak}

\usepackage{titlesec}

\setcounter{secnumdepth}{4}

\titleformat{\paragraph}
{\normalfont\normalsize\bfseries}{\theparagraph}{1em}{}
\titlespacing*{\paragraph}
{0pt}{3.25ex plus 1ex minus .2ex}{1.5ex plus .2ex}

\usepackage{multicol}
\setlength{\columnsep}{1cm}

\usepackage[most]{tcolorbox}

\newtcolorbox{leftBox}{
  colback=white,colframe=black, 
  width = 0.97\linewidth,
  sharp corners = southwest,
}

\usepackage{newfloat,caption,float}
\usepackage{capt-of}

\DeclareFloatingEnvironment[
  fileext = loe,
  listname = Задача,
  name = Задача.,
  placement = H,
  within = none,
  ]{application}
\captionsetup[application]{labelfont=md}

\DeclareFloatingEnvironment[
  fileext = loe,
  listname = Рис,
  name = Рис.,
  placement = H,
  within = none,
  ]{image}
\captionsetup[image]{labelfont=md}

\usepackage{tikz,tikz-3dplot}

\renewcommand{\lstlistingname}{Листинг}% Listing -> Листинг


\newcommand*{\task}[4]{
  \begin{minipage}[t]{\linewidth}
  \begin{multicols}{2}
    #1\\\\
    Ответ: $#2$\\
    \includegraphics[width=0.4\textwidth]{#3}
  \end{multicols}
  \captionof{application}{Пример генерации задания с помощью #4}
\end{minipage}
}

\usepackage{setspace}

\begin{document}


\begin{center}
	\hfill \break
	\large{МИНОБРНАУКИ РОССИИ}\\
	\footnotesize{ФЕДЕРАЛЬНОЕ ГОСУДАРСТВЕННОЕ БЮДЖЕТНОЕ ОБРАЗОВАТЕЛЬНОЕ УЧЕРЕЖДЕНИЕ}\\
	\footnotesize{ВЫСШЕГО ПРОФЕССИОНАЛЬНОГО ОБРАЗОВАНИЯ}\\
	\small{\textbf{«ВОРОНЕЖСКИЙ ГОСУДАРСТВЕННЫЙ УНИВЕРСИТЕТ»}}\\
	\hfill \break
	\normalsize{Факультет компьютерных наук}\\
	\hfill \break
	\normalsize{Кафедра цифровых технологий}\\
	\hfill\break
	\hfill \break
	\hfill \break
	\hfill \break
	\large{Программная реализация (на языке JavaScript) алгоритмов генерации ФОС ЕГЭ по геометрии в 2025 году}\\
	\hfill \break
	\hfill \break
	\hfill \break
	\hfill \break
	\hfill \break
	\normalsize{Курсовая работа\\
		\hfill \break
		Направление  020301 Математика и компьютерные науки\\

		\hfill \break
	}\\
	\hfill \break
	\hfill \break
\end{center}
\hfill \break

\normalsize{
	\begin{tabular}{cccc}
		Зав.кафедрой & \underline{\hspace{3cm}} & док.физ.-мат.н.,  проф. & С.Д. Кургалин    \\\\
		Обучающийся  & \underline{\hspace{3cm}} &                       & Е.Ю. Колесникова \\\\
		Руководитель & \underline{\hspace{3cm}} & доц.кан.физ-мат.н,  проф. & Н.П Стадная    \\\\
	\end{tabular}
}\\
\hfill \break
\hfill \break
\begin{center} Воронеж 2025 \end{center}
\thispagestyle{empty} % выключаем отображение номера для этой страницы

% КОНЕЦ ТИТУЛЬНОГО ЛИСТА

\setstretch{1.5}

\tableofcontents


\section*{Введение}
\addcontentsline{toc}{section}{Введение}
Единый государственный экзамен (ЕГЭ)~— централизованно проводимый в Российской
Федерации экзамен в средних учебных заведениях — школах, лицеях и гимназиях,
форма проведения ГИА (Государственной Итоговой Аттестации) по образовательным программам среднего общего образования.
Служит одновременно выпускным экзаменом из школы и вступительным экзаменом в вузы.

Но за время обучения в 9, 10 и 11 классе при подготовке к ОГЭ и ЕГЭ школьники сталкиваются с дефицитом заданий по определённым категориям.
Так, за последние 5 лет в список заданий ЕГЭ были добавлены новые задания под номером 1 по теме «Округление с недостатком» и «Округление с избытком», так же задания под номером 15 «Проценты и округление»,21 задания « Текстовые задачи», количество которых для прорешивания было мало. 
Ко всему прочему в задании номер 7 по теме «Числовые неравенства, координатная прямая -числа на прямой» банк заданий расходуется при подготовке с невероятной скоростью:
так как это преимущественно графические задания, решение их занимает менее минуты, а их составление вручную занимает несоразмерно много времени. ОГЭ и ЕГЭ является относительно неизменяемым экзаменом, поэтому все материалы, которые уже были выложены в открытый доступ, имеют полные решения, что приводит к списыванию учениками.

При этом существуют задания со вспомогательным чертежом. Чаще всего для целого ряда заданий используется одна и та же иллюстрация, которая не всегда соответствуют условиям задачи, а иногда отвлекает от решения.
Проект «Час ЕГЭ» позволяет решить все эти проблемы.

(«Час ЕГЭ»)~\cite{chas-ege} — компьютерный образовательный OpenSource-проект, предназначенный для помощи учащимся
старших классов при подготовке к тестовой части единого государственного экзамена.

Задания в «Час ЕГЭ» генерируются случайным образом по специализированным алгоритмам,
называемых шаблонами, каждый из которых
охватывает множество вариантов соответствующей ему задачи. Для
пользователей
предназначены четыре оболочки (режима работы): «Случайное задание», «Тесты на печать»,
«Полный тест» и «Мини-интеграция».
«Час ЕГЭ» является полностью открытым (код находится под лицензией GNU GPL 3.0)
и бесплатным.
В настоящее время в проекте полностью реализованы тесты по ЕГЭ по математике профильного уровня с кратким
ответом (бывшая «часть В»).~\cite{fipi}
Планируется с течением времени включить в проект тесты по другим предметам школьной
программы.

Первую главу посвятим обзору шаблонов для номеров (21, 15, 1 из ЕГЭ базовой)~\cite{egemath}.
Во второй главе рассмотрим функцию добавленному для упрощения отрисовки прямых для (7 задания ОГЭ)~\cite{ogemath}.


%Цели
%всп функции для иллюстрирования
%шаблоны

%Задачи
%всп функции для иллюстрирования
%классы для 
%шаблоны



%%Программная реализация (на языке Javascript) алгоритмов
%% генерации фонда оценочных средств по математике

\section{Планиметрия}
В это главе мы приводим вспомогательные функции и алгоритм написания шаблона по планиметрии. И рассказываем о элементах декларативного программирования в проекте.
\subsection{Вспомогательные функции}
\subsubsection{Функции для работы с массивами}
\prototype[repeat]{Array}{permuteCyclic}
Возвращает массив после циклической перестановки.
В листинге \ref{lst:92} в строке \ref{line:permuteCyclic-1} функция используется для перестановки букв в названии параллелограмма, а в строке \ref{line:permuteCyclic-2} — в названии трапеции.

\begin{lstlisting}
    let array = [1, 2, 3, 4, 5];

    array.permuteCyclic(1);
    // [5, 1, 2, 3, 4]

    array.permuteCyclic(-2);
    // [3, 4, 5, 1, 2]

    array.permuteCyclic(0);
    // [1, 2, 3, 4, 5]
    \end{lstlisting}

\prototype{Array}{mt\_coordinatesOfIntersectionOfTwoSegments\\}
Возвращает координаты пересечения двух отрезков, задаваемых первыми парами точек из массива. Является вспомогательной для функции \texttt{arcBetweenSegments}.

\begin{lstlisting}[escapechar=|]
    let array = [{x:0,y:5},{x:-4,y:4},{x:1,y:10},{x:-3,y:6}];

    array.mt_coordinatesOfIntersectionOfTwoSegments()
    //{ x: -5.333333333333333, y: 3.666666666666667, status: false }
    //Если status — false, отрезки не пересекаются, но прямые проходящие через них пересекаются в точке {x,y}

    array = [{x:0,y:5},{x:-4,y:4},{x:1,y:1},{x:-3,y:6}];
    array.mt_coordinatesOfIntersectionOfTwoSegments()
    //{ x: -1.8333333333333333, y: 4.541666666666667, status: true }
    //Если status — true, отрезки пересекаются в точке {x,y}
        
\end{lstlisting}

\prototype[separator]{Array}{shuffleJoin}
Перемешивает и соединяет массив с разделителем \texttt{separator}. \texttt{separator} по умолчанию пустая строка. Функция используется в листинге \ref{lst:3011}
в строке \ref{line:shuffleJoin} для отображения условий задачи в случайном порядке.

\begin{lstlisting}
    let array = ['A', 'B', 'C', 'D',];
    array.shuffleJoin();
    //ADBC

    array.shuffleJoin('; ');
    //C; D; B; A 
\end{lstlisting}

\prototype[separator]{Array}{joinWithConjunction}
Соединяет массив запятыми и соединяет два последних элемента союзом «и».

\begin{lstlisting}
    let array = ['A', 'B', 'C', 'D',];

    array.joinWithConjunction();
    //A, B, C и D
\end{lstlisting}

\subsubsection{Функции для работы с числами}
\prototype{Number}{perfectCubicMultiplier}
Возвращает максимальный делитель данного числа, куб которого также является делителем данного числа.

\begin{lstlisting}
    let number = 81;

    number.perfectCubicMultiplier()
    //3

    number = 36;
    number.perfectCubicMultiplier()
    //1

    number = -27;
    number.perfectCubicMultiplier()
    //3
\end{lstlisting}

\prototype[p1, p2]{Number}{texcbrt}
TeX-представление кубического корня из данного числа.\\
Если данное число - полный куб, то корень из числа.\\
Если \texttt{p1}, то из-под корня будут вынесены возможные множители.\\
Если \texttt{p1}, \texttt{p2} и из-под корня выносится единица, то она будет опущена\\
\includegraphics[width=0.8\textwidth]{texcbrt.png}

\subsubsection{Функции для работы с canvas}
\prototype[vertex,\\ fillStyle]{CanvasRenderingContext2D}{drawSection}
Заполняет область цветом \texttt{fillStyle} по вершинам из массива \texttt{vertex}.

\begin{lstlisting}
    let paint1 = function(ctx) {
        let h = 400;
        let w = 400;
        ctx.drawCoordinatePlane(w, h, {
            hor: 1,
            ver: 1
        }, {
            x1: '1',
            y1: '1',
            sh1: 16,
        },30);
        ctx.scale(30, -30);
        ctx.drawSection([[1, 3],[-3, 0],[-2, -2],[1, -1],[5, 1],[4, 4],[3, 2]]);

        ctx.drawSection([[-2, 0],[-1, 1],[-4, 3],[-1, 5],[1, 1],[5, 2],[4, -6],[0, 0],[-4, -2],]);
    };
    \end{lstlisting}
    \includegraphics[width=0.4\textwidth]{drawSection-1.png}
    \includegraphics[width=0.4\textwidth]{drawSection-2.png}

\prototype[x, y,\\ angle, length]{CanvasRenderingContext2D}{drawLineAtAngle}
Рисует отрезок длины \texttt{length} под углом angle (в радианах). Пример использования можно найти в листинге \ref{lst:2069} в строках \ref{line:drawLineAtAngle-1} и \ref{line:drawLineAtAngle-2} (применяется для отрисовки биссектрисы). 

\prototype[x1, y1, x2, y2, length, quantity]{CanvasRenderingContext2D}{strokeInMiddleOfSegment\\}
Ставит штрихи длины \texttt{length} на середине отрезка перпендикулярно ему. Функция используется в листинге \ref{lst:2069} в строках \ref{line:strokeInMiddleOfSegment-1}-\ref{line:strokeInMiddleOfSegment-2} для обозначения равных по длине сторон треугольника.

\prototype[x, y, angle, letter, length, maxLength]{CanvasRenderingContext2D}{markSegmentWithLetter\\}
Вспомогательная функция для отрисовки текста около некоторого отрезка.

\prototype[x1, y1, x2, y2, letter, length, maxLength]{CanvasRenderingContext2D}{signSegmentInMiddle\\}
Рисует строку \texttt{letter} на середине отрезка. В листинге \ref{lst:27193} в строках \ref{line:signSegmentInMiddle-1} - \ref{line:signSegmentInMiddle-2} функция используется для отображения длин рёбер многогранника.

\prototype[coordinates, radius]{CanvasRenderingContext2D}{arcBetweenSegments\\}
Рисует знак угла между двумя отрезками в месте их пересечения. \texttt{coordinates} - массив вида \texttt{[x1, y1, x2, y2]}.

\begin{lstlisting}
    let paint1 = function(ctx) {
        let h = 400;
        let w = 400;
        ctx.drawCoordinatePlane(w, h, {
            hor: 1,
            ver: 1
        }, {
            x1: '1',
            y1: '1',
            sh1: 16,
        }, 30);
        ctx.scale(30, -30);

        ctx.lineWidth = 2 / 30;
        ctx.drawLine(1, 5, 3, -2);
        ctx.drawLine(3, -2, 5, -3);
        ctx.arcBetweenSegments([1, 5, 3, -2, 5, -3], 2);

        ctx.drawLine(2, -5, -4, -2);
        ctx.drawLine(1, -2, -3, -6);
        ctx.arcBetweenSegments([2, -5, -4, -2,  -3, -6,1, -2,], 1);

        ctx.drawLine(1, 5, 1, -2);
		    ctx.drawLine(1, -2, 5, -2);
		    ctx.strokeStyle = om.secondaryBrandColors.iz();
		    ctx.arcBetweenSegments([1, 5, 1, -2, 5, -2], 3);

    };
\end{lstlisting}

\includegraphics[width=0.4\textwidth]{arcBetweenSegments-1.png}    
\includegraphics[width=0.4\textwidth]{arcBetweenSegments-2.png}    

\prototype[coordinates, radius, number, step]{CanvasRenderingContext2D}{arcBetweenSegmentsCount\\}
Рисует знак угла между двумя отрезками в месте их пересечения \texttt{number} раз с отступом \texttt{step}. В листинге \ref{lst:27764} в строках \ref{line:arcBetweenSegmentsCount-1} - \ref{line:arcBetweenSegmentsCount-2} используется для обозначения двух равных углов.

\prototype[x, y, radiusX, radiusY, rotation, startAngle, endAngle,\\ anticlockwise]{CanvasRenderingContext2D}{drawEllipse\\}
Рисует эллипс.

\prototype[x, y, radius, startAngle, endAngle, anticlockwise]{CanvasRenderingContext2D}{drawArc\\}
Рисует дугу.

\subsubsection{Элементы декларативного программирования}

\textbf{Определение.} Декларативное программирование — парадигма программирования, в которой задается спецификация решения задачи, то есть описывается конечный результат, а не способ его достижения.~\cite{posobie}

Во время разработки шаблонов по теме «Графики функции» требовалось много раз генерировать коэффициенты функций через циклы 
\texttt{while} или \texttt{do\dots while}, пока они не начнут соответствовать заданным условиям (видимость графика, сливание его с осями, видимость целых точек). Это часто приводило к бесконечной работе шаблона, при этом сложно было определить, какое условие не выполняется.

Для этого было разработано окружение \texttt{retryWhileUndefined} для шаблонов, которое бы перезапускало их не более \texttt{maxIterations} раз, если одно из условий не удовлетворено. 

\function{retryWhileUndefined}{theFunction, maxIterations}

Но всё равно было тяжело определить, почему шаблон перезапускается. Для этого было разработано более совершенное окружение \texttt{retryWhileError}, которое не только могло бы ограничивать количество перезапусков, но и фиксировать, какие проверки не были пройдены и выводить их на экран (ошибки видны только для разработчика при отладке).

\function{retryWhileError}{theFunction, maxIterations,maxCollectedErrors}

Для окружения были написаны функции-утверждения, которые имеют структуру: условие не выполнено - записать ошибку - перезапустить шаблон. Если максимальное количество повторений достигнуто, то вывести накопившиеся ошибки и количество их появлений. 

\function{genAssert}{condition, message}
    Если условие \texttt{condition} ложно, то шаблон перезапускается. 

\function{genAssertNonempty}{array, message}
    Если массив \texttt{array} пуст, то шаблон перезапускается.

\function{genAssertZ1000}{number, message}
    Если число \texttt{number} имеет более 3 знаков после запятой, то шаблон перезапускается.
    
\function{genAssertIrreducible}{numerator, denominator, message}
    Если дробь \texttt{numerator/denominator} сократима, то шаблон перезапускается.

\function{genAssertSaneDecomposition}{number, maxFactor, message}
    Если \texttt{number} число не раскладывается на простые множители, не более одного из которых превосходит \texttt{maxFactor}, то шаблон перезапускается. 

\subsection{Этапы разработки шаблона со вспомогательным чертежом по теме «Планиметрия»}

Для примера возьмём задание №19416~\cite{egemath}.
\\
\textbf{Задача №19376.}
В треугольнике $ABC$ известно, что ${AC=BC}$, $AB=16$, $AH$ — высота, $BH=4$. Найдите косинус угла $BAC$.\\ 

Заготовка шаблона имеет вид.

\lstinputlisting[]{code/109_1.js}

\begin{enumerate}
    \item Начнём с отрисовки чертежа для задания. Отметим стороны треугольника так, чтобы он лежал в центре холста, а до краёв оставалось 10-20px. При отрисовке используем функцию \texttt{drawLine}. Добавим высоту 
    \lstinputlisting[]{code/109_2.js} 
    \item Добавим на рисунок штрихи, указывающие на равенство сторон и обозначение прямого угла при помощи функций \texttt{strokeInMiddleOfSegment} и arcBetweenSegments соответственно. И подпишем вершины и точку пересечения высоты и основания. Добавим модификатор \texttt{NAtask.modifiers.variativeABC(vertices)}, который заменяет все буквы в задании на случайные.
   

    \lstinputlisting[]{code/109_3.js} 
    \item Теперь добавим ответ в задание. Проверим при помощи \texttt{genAssertZ1000}, что ответ имеет не более трёх знаков после запятой (иначе шаблон запускается заново). Поместим все буквы и числа в \$\dots\$. Все условия из задачи преобразуем в массив и соединим случайным образом с помощью функции \texttt{shuffleJoin}.
    \lstinputlisting[]{code/109_4.js} 
\end{enumerate}

Примеры генерации задний приведены в листинге \ref{lst:109}
 %TODO: Приложение зафигачить


\section{Генерация текстовых задач}

В Первом разделе представлены работы, связанные Текстовыми задачами. 
%Работа велась на языке \texttt{JavaScript}, с использованием вспомогательных функций для визуализации и генерации условий.
Так как в 
%15, 21 "текстовые задачи"
%рассказать про декорации, про chislitlx(пулл реквест с r2), сочетания 
%предварительные сведения (функции, кратко ,генассерт и тп)
%платформа час-ЕГЭ (то что сделала сама "хвалюсь") для "реализации задачи мною были разработаны" и тп, 
% то что существует - описываю как факт (а.к.а предварительные сведения)
% 4 1)отдельно (формулы) естественно появляется обратная задача. 2)Сослаться на Латех (MathJax) основные команды которыми пользуемся в Латехе texfrag и тп 
%7 делится на две главы. Задача с чертежом коорд.оси (предвор сведения, разработанные библ функции, шаблоны)
% вторая - задача на сравнение чисел (как устроено округление,autoLateX )
% в заключении написать количество вып задач по категориям 
%в данной главе (тото это тамто и тп)
% не менее 10 шаблонов сумарно затронув каждую категорию 
% где-то 45-50 страниц....
% про расвнение чисел очень долго целых чисел не было, все числа были дробные, 0.1+0.2!=0.3 надо быть осторожным и тп расскажи!! 
%захожу в калатог, через меню переключаемся на тест на печать внимательно (?) выставить нулями все задания кроме того что нужно ставить галочку " экспорт в латех" 
%ставим запуск,жду,думаю, падает зипка с задачами в латехе, в перемешку 
%исходная задача/код/полученная задача (сгенерин текст)
При построении текстовых шаблонов задач важно обеспечить как вариативность формулировок, так и корректность числовых и языковых выражений. Для этого в проекте применяются следующие инструменты.

\subsection{Декорации}

Чтобы увеличить количество уникальных вариантов задач, сохраняя их математическую суть, используются так называемые ``декорации''~--- элементы окружения, которые можно менять без потери смысла задачи: имена персонажей, место действия, цель, контекст. 

Для этого создаются массивы строк, а функция \texttt{iz()} случайным образом выбирает элемент из массива.  
Пример:
\begin{lstlisting}
let contract = ['о дружбе', 'во избежание двойного налогообложения',
    'о безвизовом режиме', 'об экологической среде',
    'по гуманитарным вопросам', 'по вопросам безопасности'].iz();
\end{lstlisting}
\textbf{Задача №514913.}

Если необходимо выбрать два случайных, но не повторяющихся элемента, используется форма \texttt{iz(2)}:  
\begin{lstlisting}
let tapeName = sklonlxkand(['лента', 'верёвка', 'нитка'].iz(2));
\end{lstlisting}
\textbf{Задача №2434.}

\subsection{Склонение существительных}

В силу особенностей русского языка слова должны корректно изменяться по падежам и числам. Для этого применяется функция \texttt{sklonlxkand}, позволяющая генерировать правильные словоформы.  

После выбора слова к нему можно обратиться по падежу и числу. Например, \texttt{.ie} означает именительный падеж, единственное число.  

Пример использования:
\begin{lstlisting}
let typeOfFlowerInVases =
  sklonlxkand(['роза','гвоздника','ромашка','лилия',
               'мак','ирис','лаванда','мимоза'].iz());

NAtask.setTask({
  text: 'На прилавке цветочного магазина стоят три вазы с '
    + typeOfFlowerInVases.tm + ': ' + vaseColor[0] + 'ая, '
    + vaseColor[1] + 'ая, ' + vaseColor[2] + 'ая.' 
    + ' Слева от ' + vaseColor[2] + 'ой вазы '
    + chislitlx(leftOfThirdVase, typeOfFlowerInVases, '$')
    + ', справа от ' + vaseColor[0] + 'ой вазы '
    + chislitlx(righttOfFirstVase, typeOfFlowerInVases, '$') + '. '
    + 'Всего в вазах '
    + chislitlx(allFlowerInVases, typeOfFlowerInVases, '$') + '. '
    + 'Сколько ' + typeOfFlowerInVases.rm + ' в '
    + vaseColor[1] + 'ой вазе?',
  answers: secondVaseCountFlower,
});
\end{lstlisting}
\textbf{Задача №515842.}

\subsection{Связка числа и существительного}

Для корректного согласования числительных с существительными используется функция \texttt{chislitlx}. Она автоматически выбирает нужную форму слова в зависимости от числа.  
Пример:
\begin{lstlisting}
let numberOfApartamentPerFloor = sluchch(3, 12, 1);
NAtask.setTask({
  text: 'В доме, в котором живет ' + nameOfPerson + ', ' +
    '$' + floorNumber + '$ этажей и несколько подъездов. На каждом этаже находится по ' +
    chislitlx(numberOfApartamentPerFloor, 'квартира', '$') + '. '
    + nameOfPerson + ' живет в квартире №' + '$' + apartamentNumber + '$' + '. '
    + 'В каком подъезде живет ' + nameOfPerson + '? ',
  answers: '$' + (apartamentNumber /
             (floorNumber * numberOfApartamentPerFloor)).ceil() + '$',
});
\end{lstlisting}
\textbf{Задача №77351.}

\subsection{Проверка условий (генерация утверждений)}
Для контроля корректности задачи применяются функции \texttt{genAssert} и её вариации. Принцип работы:  
\begin{itemize}
    \item если условие не выполнено~--- фиксируется ошибка;
    \item шаблон перезапускается;
    \item если достигнуто максимальное количество перезапусков, выводятся накопившиеся ошибки с указанием количества.
\end{itemize}

Пример использования:
\begin{lstlisting}
let students = sl(200, 10000, 10);
let percent = sl(10, 90, 1); 
genAssert(students.kratno(100 / percent),
          "количество учащихся не кратно 100/процент");
\end{lstlisting}
\textbf{Задача №77340.}

Также часто используется функция \function{genAssertZ1000}{number, message}, проверяющая точность чисел: если у числа более трёх знаков после запятой, шаблон перезапускается.  
Пример:
\begin{lstlisting}
let prise = sl(100, 2000, 10);
let percentSecondMonth = sl(10, 90, 1);
let percentThirdMonth = sl(10, 90, 1);

let middlePrise = prise * (1 + [1, -1][randFirst] * 0.01 * percentSecondMonth);
let finalePrise = middlePrise * (1 + [1, -1][randSecond] * 0.01 * percentThirdMonth);

genAssertZ1000(finalePrise / 10,
               'Число имеет более 2 знаков после запятой');
\end{lstlisting}
\textbf{Задача №77349.}

\include{part_2.tex}

\section*{Заключение}
\addcontentsline{toc}{section}{Заключение}
В ходе выполнеия курсовой работы за 3 курс был покрыт открытый банк заданий ФИПИ по темам:
		      \begin{itemize}
			      \item Текстовые задачи (на смекалку) — 12 шаблонов принято.
			      \item Текстовые задачи (проценты и дроби) — 29 шаблонов принято.
			      \item Преобразования выражений — 29 шаблонов (25 принято 4 на внутреннем рецензировании).
			      \item Задачи с прямыми — 10 шаблонов принято. (3 с рисунком и 7 на сравнение чисел )
		      \end{itemize}

В ядро проекта добавлены: 
\begin{itemize}
    \item Функции, упрощающие написание шаблонов по теме «Координатная прямая».
    \item r2 % как это описать, как это описать....
\end{itemize}

А также сокращён технический долг проекта.

Все добавленные в проект задания можно использовать для составления контрольных работ, проведения текущего контроля знаний учащихся, подготовки к ЕГЭ.~\cite{chas-ege}

В будущем планируется добавить% в проект класс плоских геометрических фигур и использовать в заданиях по теме «Планиметрия» динамические изображения.




\begin{thebibliography}{6}
	\bibitem{chas-ege} Тренажёр "Час ЕГЭ". – URL: https://math.vsu.ru/chas-ege/sh/katalog.html
	\bibitem{fipi}Федеральный институт педагогических измерений. – URL:  https://fipi.ru/ege/otkrytyy-bank-zadaniy-ege
	\bibitem{egemath}Открытый банк задач ОГЭ по математике. – URL:  https://oge.sdamgia.ru/?Redir=1
	\bibitem{egemath}Открытый банк задач ЕГЭ по математике. Базовый уровень. – URL:  https://mathb-ege.sdamgia.ru
	\bibitem{ege} Единый государственный экзамен. – URL:  https://ru.wikipedia.org/wiki/Единый\_государственный\_экзамен
	\bibitem{sdamgia}Решу ЕГЭ - Сдам ГИА. – URL: https://ege.sdamgia.ru/problem?id=27074
\end{thebibliography}

\include{applications.tex}

\end{document}

%Планиметрия
  %Вспомогательные функции
  %шаблоны
%Нейронные сети и шаблоны по теме «Стереометрия»
  %Современный прогресс генеративных текстовых нейросетей
  %Разработка библиотек с помощью Gpt-Chat
  %Применение ООП для разработки шаблонов
  %шаблоны

%Введение
%определение шаблона
%Основные сведения о проекте
  %1) используемые технологи
  %2)Внутренние библиотеки
  %3)Примеры шаблонов заданий(мои старые простые)
%Шаблоны на чтение графиков функции с произвольным параметром
  %вспомогательные функции добавленные к библиотеке
  %разработанные шаблоны
%Сплайны третьего порядка и шаблоны на чтение графика функции и её производной
  %понятие сплайна 3 порядка + про библиотеку
  %вспомогательные функцииЫ
  %шаблоны
%ВОЗМОЖНО ТОЛЬКО ЭТО

\documentclass[a4paper, 14pt]{extarticle}

\usepackage{ifluatex}
\usepackage{ifpdf}

\ifluatex
\usepackage{fontspec}
\defaultfontfeatures{Renderer=Basic,Ligatures={TeX}}
\setmainfont{CMU Serif}
\setsansfont{CMU Sans Serif}
\usepackage{polyglossia}
\setdefaultlanguage{russian}
\setotherlanguage{english}
\setmainfont{CMU Serif}
\newfontfamily{\cyrillicfont}{CMU Serif}
\setsansfont{CMU Sans Serif}
\newfontfamily{\cyrillicfontsf}{CMU Sans Serif}
\setmonofont{CMU Typewriter Text}
\newfontfamily{\cyrillicfonttt}{CMU Typewriter Text}
\else
\ifpdf 
\usepackage[T2A]{fontenc}
\usepackage[utf8]{inputenc}
\usepackage[english,russian]{babel}
\typeout{PDF only}
\fi
\fi

\usepackage[left=1cm,right=1cm,top=2cm,bottom=2cm]{geometry}
%%% Дополнительная работа с математикой
\usepackage{amsfonts,amssymb,amsthm,mathtools} % AMS
\usepackage{amsmath}
\usepackage{icomma} % «Умная» запятая: $0,2$ — число, $0, 2$ — перечисление

\usepackage{mathrsfs} % Красивый матшрифт

%% Перенос знаков в формулах (по Львовскому)
\newcommand*{\hm}[1]{#1\nobreak\discretionary{}
  {\hbox{$\mathsurround=0pt #1$}}{}}

%%% Работа с картинками

\usepackage{graphicx}  % Для вставки рисунков
\graphicspath{ {images/} }
\setlength\fboxsep{3pt} % Отступ рамки \fbox{} от рисунка
\setlength\fboxrule{1pt} % Толщина линий рамки \fbox{}
\usepackage{wrapfig} % Обтекание рисунков и таблиц текстом

\usepackage[dvipsnames]{xcolor}
\usepackage{verbatim}
\usepackage{hyperref}

\usepackage{listings}

\lstdefinelanguage{JavaScript}{
  keywords={let, typeof, new, true, false, catch, function, return, null, catch, switch, var, if, in, while, do, else, case, break, const},
  keywordstyle=\color{blue}\bfseries,
  ndkeywords={class, export, boolean, throw, implements, import, this,},
  ndkeywordstyle=\color{darkgray}\bfseries,
  identifierstyle=\color{black},
  sensitive=false,
  comment=[l]{//},
  morecomment=[s]{/*}{*/},
  commentstyle=\color{green}\ttfamily,
  stringstyle=\color{red}\ttfamily,
  morestring=[b]',
  morestring=[b]",
  escapechar=|
}

\lstset{
  language=JavaScript,
  extendedchars=true,
  basicstyle=\small\ttfamily,
  showstringspaces=false,
  breakatwhitespace=true,
  showspaces=false,
  numbers=left,
  numberstyle=\footnotesize,
  numbersep=9pt,
  tabsize=2,
  keepspaces=true,
  breaklines=true,
  showtabs=false,
  captionpos=b
  escapechar=|,
  frame=single ,
  commentstyle=\itshape ,
  frameshape={RYR}{Y}{Y}{RYR},
  stringstyle =\bfseries,
}

\usepackage{autobreak}

\newcommand*{\function}[2]{\texttt{\textcolor{Red}{function} \textcolor{Purple}{#1}(#2)}\linebreak}

\newcommand*{\prototype}[3][]{\texttt{\textcolor{Orange}{#2}.\textcolor{Blue}{prototype}.\textcolor{Purple}{#3} = \textcolor{Red}{function}(#1)}\linebreak}

\usepackage{titlesec}

\setcounter{secnumdepth}{4}

\titleformat{\paragraph}
{\normalfont\normalsize\bfseries}{\theparagraph}{1em}{}
\titlespacing*{\paragraph}
{0pt}{3.25ex plus 1ex minus .2ex}{1.5ex plus .2ex}

\usepackage{multicol}
\setlength{\columnsep}{1cm}

\usepackage[most]{tcolorbox}

\newtcolorbox{leftBox}{
  colback=white,colframe=black, 
  width = 0.97\linewidth,
  sharp corners = southwest,
}

\usepackage{newfloat,caption,float}
\usepackage{capt-of}

\DeclareFloatingEnvironment[
  fileext = loe,
  listname = Задача,
  name = Задача.,
  placement = H,
  within = none,
  ]{application}
\captionsetup[application]{labelfont=md}

\DeclareFloatingEnvironment[
  fileext = loe,
  listname = Рис,
  name = Рис.,
  placement = H,
  within = none,
  ]{image}
\captionsetup[image]{labelfont=md}

\usepackage{tikz,tikz-3dplot}

\renewcommand{\lstlistingname}{Листинг}% Listing -> Листинг


\newcommand*{\task}[4]{
  \begin{minipage}[t]{\linewidth}
  \begin{multicols}{2}
    #1\\\\
    Ответ: $#2$\\
    \includegraphics[width=0.4\textwidth]{#3}
  \end{multicols}
  \captionof{application}{Пример генерации задания с помощью #4}
\end{minipage}
}

\usepackage{setspace}

\begin{document}


\begin{center}
	\hfill \break
	\large{МИНОБРНАУКИ РОССИИ}\\
	\footnotesize{ФЕДЕРАЛЬНОЕ ГОСУДАРСТВЕННОЕ БЮДЖЕТНОЕ ОБРАЗОВАТЕЛЬНОЕ УЧЕРЕЖДЕНИЕ}\\
	\footnotesize{ВЫСШЕГО ПРОФЕССИОНАЛЬНОГО ОБРАЗОВАНИЯ}\\
	\small{\textbf{«ВОРОНЕЖСКИЙ ГОСУДАРСТВЕННЫЙ УНИВЕРСИТЕТ»}}\\
	\hfill \break
	\normalsize{Факультет компьютерных наук}\\
	\hfill \break
	\normalsize{Кафедра цифровых технологий}\\
	\hfill\break
	\hfill \break
	\hfill \break
	\hfill \break
	\large{Программная реализация (на языке JavaScript) алгоритмов генерации ФОС ЕГЭ по геометрии в 2025 году}\\
	\hfill \break
	\hfill \break
	\hfill \break
	\hfill \break
	\hfill \break
	\normalsize{Курсовая работа\\
		\hfill \break
		Направление  020301 Математика и компьютерные науки\\

		\hfill \break
	}\\
	\hfill \break
	\hfill \break
\end{center}
\hfill \break

\normalsize{
	\begin{tabular}{cccc}
		Зав.кафедрой & \underline{\hspace{3cm}} & д.физ.-мат.н.,  проф. & С.Д. Кургалин    \\\\
		Обучающийся  & \underline{\hspace{3cm}} &                       & Е.Ю. Колесникова \\\\
		Руководитель & \underline{\hspace{3cm}} & к.физ-мат.н,      доц. & Н.П. Стадная    \\\\
	\end{tabular}
}\\
\hfill \break
\hfill \break
\begin{center} Воронеж 2025 \end{center}
\thispagestyle{empty} % выключаем отображение номера для этой страницы

% КОНЕЦ ТИТУЛЬНОГО ЛИСТА

\setstretch{1.5}

\tableofcontents


\section*{Введение}
\addcontentsline{toc}{section}{Введение}
Единый государственный экзамен (ЕГЭ)~— централизованно проводимый в Российской
Федерации экзамен в средних учебных заведениях — школах, лицеях и гимназиях,
форма проведения ГИА(Государственный Итоговая Аттестация) по образовательным программам среднего общего образования.
Служит одновременно выпускным экзаменом из школы и вступительным экзаменом в вузы.

Но за время обучения в 10 и 11 классе при подготовке к ЕГЭ школьники сталкиваются с дефицитом заданий по определённым категориям.
Так в конце 2021 года в список заданий ЕГЭ были добавлены новые задания под номером 11 по теме «Графики функций», а в конце 2023 — задание №2 по теме «Вектора», количество которых для прорешивания было очень мало. 
А по теме «Производная и первообразная» банк заданий расходуется при подготовке с невероятной скоростью:
так как это преимущественно графические задания, решение их занимает менее минуты, а их составление вручную занимает несоразмерно много времени.

ЕГЭ является относительно неизменяемым экзаменом, поэтому все материалы, которые уже были выложены в открытый доступ, имеют полные решения, что приводят к списыванию учениками.

При этом существуют задания с вспомогательным чертежом. Чаще всего для целого ряда заданий используется одна и та же иллюстрация, которая не всегда соответствуют условиям задачи, а иногда отвлекает от решения.
Проект «Час ЕГЭ» позволяет решить все эти проблемы.

«Час ЕГЭ» — компьютерный образовательный проект, разрабатываемый при математическом
факультете ВГУ в рамках «OpenSource кластера» и предназначенный для помощи учащимся
старших классов при подготовке к тестовой части единого государственного экзамена.
%%ссылочки на доклады
Задания в «Час ЕГЭ» генерируются случайным образом по специализированным алгоритмам,
называемых шаблонами, каждый из которых
охватывает множество вариантов соответствующей ему задачи. Для
пользователей
предназначены четыре оболочки (режима работы): «Случайное задание», «Тесты на печать»,
«Полный тест» и «Мини-интеграция».
«Час ЕГЭ» является полностью открытым (код находится под лицензией GNU GPL 3.0)
и бесплатным.
В настоящее время в проекте полностью реализованы тесты по математике с кратким
ответом (бывшая «часть В»).~\cite{fipi}
Планируется с течением времени включить в проект тесты по другим предметам школьной
программы.

Первая глава этой работы посвящена обзору вспомогательных функций, которые ускоряют написание шаблонов по теме «Планиметрия» и введению в проект элементов декларативного программирования. Также приведён алгоритм написания шаблона с чертежом.

Вторая глава представляет решение проблемы отрисовки фигур в трёхмерном пространстве на языке программирования JavaScript; рассказывает о применении объектно-ориентированного программирования для упрощения написания шаблонов с чертежом; затрагивает вопрос об написании программного кода при помощи нейросетей; приводит обзор вспомогательных функций и алгоритм написания шаблона по теме «Стереометрия». 


%Цели
%всп функции для иллюстрирования
%шаблоны

%Задачи
%всп функции для иллюстрирования
%классы для 
%шаблоны



%%Программная реализация (на языке Javascript) алгоритмов
%% генерации фонда оценочных средств по математике
\section{Глава первая}


\section{Стереометрия}\label{2sect}
\subsection{Разработка библиотек с помощью Gpt-Chat}

На данный момент в языке JavaScript отсутствуют встроенные средства для изображения трёхмерных фигур. И существует только одна подходящая библиотека \texttt{Three.js}, которая могла бы выполнить проецирование координат фигуры на плоскость с учётом положения наблюдателя. Но при этом для создания любого объекта необходима не только камера, но и сцена с рендерингом, что значительно замедляет работу проекта.

\begin{lstlisting}[caption={Код, необходыимый для отрисовки куба}]
import * as THREE from 'three';

// Создать сцену, камеру и рендер
const scene = new THREE.Scene();
const camera = new THREE.PerspectiveCamera(75, window.innerWidth / window.innerHeight, 0.1, 1000);
camera.position.z = 2;
const renderer = new THREE.WebGLRenderer();
renderer.setSize(window.innerWidth, window.innerHeight);

// Создать геометрию куба
const geometry =new THREE.BoxGeometry(1, 1, 1);

// Создать материал куба
const material = new THREE.MeshBasicMaterial({ color: 0x00ff00 });

// Создать куб
const cube = new THREE.Mesh(geometry, material);

// Добавить куб в сцену
scene.add(cube);

// Отрендерить сцену
renderer.render(scene, camera);

\end{lstlisting}

%Заменить на: увеличивает размер проекта. Про сложность вызова возможно. Обосновать почему не подходит. Возможно приложить пример использования. 

Другие подобные ей библиотеки проводят проецирование на плоскость с поворотом только вокруг осей $Ox$ и $Oz$, что при отрисовке более сложных чертежей является недостаточным.

С учётом прогресса современных технологий встал закономерный вопрос, сможет ли нейросеть ChatGPT 3.5 сгенерировать код проектора. Ниже представлен диалог с нейросетью:

\begin{leftBox}
	write a 2D to 3D projector in JS without using libraries. input:  point = \{x,y,z\}, camera= \{x,y,z, angleX, angleY, angleZ, scale\}
\end{leftBox}

\begin{lstlisting}[numbers=none]
	
	// Define the point in 2D
	const point = { x: 100, y: 50, z: 0 };

	// Define the camera parameters
	const camera = { x: 0, y: 0, z: 0, angleX: 0, angleY: 0, angleZ: 0, scale: 1 };

	// Function to project 2D point to 3D
	function project2Dto3D(point, camera) {
			// Apply rotation around X axis
			const rotatedY = point.y * Math.cos(camera.angleX) - point.z * Math.sin(camera.angleX);
			const rotatedZ = point.y * Math.sin(camera.angleX) + point.z * Math.cos(camera.angleX);

			// Apply rotation around Y axis
			const rotatedX = point.x * Math.cos(camera.angleY) + rotatedZ * Math.sin(camera.angleY);
			const rotatedZ2 = -point.x * Math.sin(camera.angleY) + rotatedZ * Math.cos(camera.angleY);

			// Apply rotation around Z axis
			const rotatedX2 = rotatedX * Math.cos(camera.angleZ) - rotatedY * Math.sin(camera.angleZ);
			const rotatedY2 = rotatedX * Math.sin(camera.angleZ) + rotatedY * Math.cos(camera.angleZ);

			// Apply scaling
			const scaledX = rotatedX2 * camera.scale;
			const scaledY = rotatedY2 * camera.scale;

			// Translate to camera position
			const translatedX = scaledX + camera.x;
			const translatedY = scaledY + camera.y;
			const translatedZ = rotatedZ2 * camera.scale + camera.z;

			return { x: translatedX, y: translatedY, z: translatedZ };
		}

	// Project the point from 2D to 3D
	const projectedPoint = project2Dto3D(point, camera);

	// Output the result
	console.log(»Projected 3D point:», projectedPoint);
	
\end{lstlisting}

\begin{leftBox}
	shorten function project2Dto3D
\end{leftBox}

\begin{lstlisting}[numbers=none]
	
	function project2Dto3D(point, camera) {
			const { x, y, z } = point;
			const { angleX, angleY, angleZ, scale, x: cx, y: cy, z: cz } = camera;

			const cosX = Math.cos(angleX);
			const sinX = Math.sin(angleX);
			const cosY = Math.cos(angleY);
			const sinY = Math.sin(angleY);
			const cosZ = Math.cos(angleZ);
			const sinZ = Math.sin(angleZ);

			const rx = x * cosY * cosZ - y * cosY * sinZ + z * sinY;
			const ry = (x * (sinX * sinY * cosZ + cosX * sinZ) + y * (sinX * sinY * sinZ - cosX * cosZ) + z * sinX * cosY) * scale;
			const rz = (-x * (cosX * sinY * cosZ - sinX * sinZ) - y * (cosX * sinY * sinZ + sinX * cosZ) + z * cosX * cosY) * scale;

			return { x: rx + cx, y: ry + cy, z: rz + cz };
		}
	
\end{lstlisting}


\begin{leftBox}
	add comments in code
\end{leftBox}

\begin{lstlisting}[numbers=none]
	
	function project2Dto3D(point, camera) {
			// Extract point coordinates
			const { x, y, z } = point;
			// Extract camera parameters
			const { angleX, angleY, angleZ, scale, x: cx, y: cy, z: cz } = camera;

			// Precompute trigonometric values for rotation angles
			const cosX = Math.cos(angleX);
			const sinX = Math.sin(angleX);
			const cosY = Math.cos(angleY);
			const sinY = Math.sin(angleY);
			const cosZ = Math.cos(angleZ);
			const sinZ = Math.sin(angleZ);

			// Apply rotation around X axis
			const rx = x * cosY * cosZ - y * cosY * sinZ + z * sinY;
			// Apply rotation around Y axis and scale
			const ry = (x * (sinX * sinY * cosZ + cosX * sinZ) + y * (sinX * sinY * sinZ - cosX * cosZ) + z * sinX * cosY) * scale;
			// Apply rotation around Z axis and scale
			const rz = (-x * (cosX * sinY * cosZ - sinX * sinZ) - y * (cosX * sinY * sinZ + sinX * cosZ) + z * cosX * cosY) * scale;

			// Translate to camera position
			const translatedX = rx + cx;
			const translatedY = ry + cy;
			const translatedZ = rz + cz;

			return { x: translatedX, y: translatedY, z: translatedZ };
		}
	
\end{lstlisting}

За несколько шагов удалось получить корректный, оптимизированный код.

\subsection{Применение объектно-ориентированного программирования для разработки шаблонов}


Банк заданий содержит большое количество разнообразных задач по теме «Стереометрия». Поэтому одной из первостепенных задач было сократить код шаблонов и исключить вычислительные ошибки. Для этого были разработаны классы многогранников, которые содержат в себе длины рёбер, объем, площади оснований, а так же тернарную матрицу связности и канонические координаты вершин.

Матрица может содержать значения: 1, 0, либо специальное значение, указывающие на отображение ребра пунктиром.

Пример канонической матрицы связей:

\begin{lstlisting}[caption = {Каноническая матрица связей для параллелепипеда}]
	[   [1],
	    [0, 1],
	    [1, 0, 1],
	    [0, 0, 0, 1],
	    [1, 0, 0, 0, 1],
	    [0, 1, 0, 0, 0, 1],
	    [0, 0, 1, 0, 1, 0, 1],
	];
\end{lstlisting}

Мы можем опускать конец матрицы, если он состоит только из нулей, и главную диагональ матрицы (на ней всегда стоят нули).

\textbf{Определение.} Каноническим положением будем называть такое расположение многогранника, когда его высота, проходящая через центр масс его основания, совпадает с осью аппликат и началом координат делится пополам (Рис.~\ref{fig:fig1}-Рис.~\ref{fig:fig2}).

При таком расположении, начало координат можно расположить в центре иллюстрации. Тогда чертёж не будет чрезмерно смещён ни в одну из сторон.
\newpage
\begin{multicols}{2}
	\begin{figure}[H]
		\tdplotsetmaincoords{75}{125}
		\begin{tikzpicture}
			[tdplot_main_coords,
				scale=0.9,
				line/.style={very thick, blue},
				linedash/.style={very thick,dashed, blue},
				diagdash/.style={very thick,dashed, blue},
				axis/.style={->,black, thick},
				axisdash/.style={thick,dashed, black},
				figure/.style={opacity=.2,very thick,fill=blue,},]
	
			\coordinate (O) at (0,0,0);
			\coordinate (A) at (-2,-2,-2);
			\coordinate (B) at (-2,2,-2);
			\coordinate (C) at (2,2,-2);
			\coordinate (D) at (2,-2,-2);
	
			\coordinate (A1) at (-2,-2,2);
			\coordinate (B1) at (-2,2,2);
			\coordinate (C1) at (2,2,2);
			\coordinate (D1) at (2,-2,2);
	
			%draw the axes
			\draw[axis] (O) -- (6,0,0) node[anchor=west]{$x$};
			\draw[axis] (O) -- (0,5,0) node[anchor=west]{$y$};
			\draw[axis] (O) -- (0,0,3) node[anchor=west]{$z$};
			\draw[axisdash] (-5,0,0) -- (O);
			\draw[axisdash] (0,-4,0) -- (O);
			\draw[axisdash] (0,0,-3) -- (O);
	
			%draw the bottom of the cube
			\draw[figure] (A) -- (B) -- (C) -- (D) -- cycle;
	
			%draw the back-right of the cube
			\draw[figure] (A) -- (B) -- (B1) -- (A1) -- cycle;
	
			%draw the back-left of the cube
			\draw[figure] (A) -- (D) -- (D1) -- (A1) -- cycle;
	
			%draw the front-right of the cube
			\draw[figure] (D) -- (C) -- (C1) -- (D1) -- cycle;
	
			%draw the front-left of the cube
			\draw[figure] (B) -- (C) -- (C1) -- (B1) -- cycle;
	
			%draw the top of the cube
			\draw[figure] (A1) -- (B1) -- (C1) -- (D1) -- cycle;
	
			\draw[diagdash] (C) -- (A);
			\draw[diagdash] (B) -- (D);
		\end{tikzpicture}
		\captionof{image}{Каноническое положение для параллелепипеда}
		\label{fig:fig1}
	\end{figure}
		\begin{figure}[H]
		\tdplotsetmaincoords{75}{115}
		\begin{tikzpicture}
			[tdplot_main_coords,
				scale=0.9,
				line/.style={very thick, blue},
				linedash/.style={very thick,dashed, blue},
				diagdash/.style={very thick,dashed, blue},
				axis/.style={->,black, thick},
				axisdash/.style={thick,dashed, black},
				figure/.style={opacity=.2,very thick,fill=blue,},]
	
			\coordinate (O) at (0,0,0);
	
			\coordinate (A) at (3,0,-2);
			\coordinate (B) at (0.9270,2.8531,-2);
			\coordinate (C) at (-2.427,1.7633,-2);
			\coordinate (D) at (-2.4270,-1.7633,-2);
			\coordinate (E) at (0.9270,-2.8531, -2);
	
			\coordinate (A1) at (3,0,2);
			\coordinate (B1) at (0.9270,2.8531,2);
			\coordinate (C1) at (-2.427,1.7633,2);
			\coordinate (D1) at (-2.4270,-1.7633,2);
			\coordinate (E1) at (0.9270,-2.8531, 2);
	
			\coordinate (A2) at (-2.43, 0, -2);
			\coordinate (B2) at (-0.75, -2.31,-2);
			\coordinate (C2) at (1.96, -1.43,-2);
			\coordinate (D2) at (1.96, 1.43,-2);
			\coordinate (E2) at (-0.75, 2.31, -2);
	
			%draw the axes
			\draw[axis] (O) -- (9,0,0) node[anchor=west]{$x$};
			\draw[axis] (O) -- (0,4,0) node[anchor=west]{$y$};
			\draw[axis] (O) -- (0,0,3) node[anchor=west]{$z$};
			\draw[axisdash] (-5,0,0) -- (O);
			\draw[axisdash] (0,-4,0) -- (O);
			\draw[axisdash] (0,0,-3) -- (O);
	
			\draw[figure] (A) -- (B)  -- (C)  -- (D) -- (E) -- cycle;
			\draw[figure] (A1) -- (B1)  -- (C1)  -- (D1) -- (E1) -- cycle;
	
			\draw[figure] (A) -- (B)  -- (B1) -- (A1) -- cycle;
			\draw[figure] (C) -- (B)  -- (B1) -- (C1) -- cycle;
			\draw[figure] (D) -- (C)  -- (C1) -- (D1) -- cycle;
			\draw[figure] (D) -- (E)  -- (E1) -- (D1) -- cycle;
			\draw[figure] (A) -- (E)  -- (E1) -- (A1) -- cycle;
	
			\draw[linedash] (A) -- (A2);
			\draw[linedash] (B) -- (B2);
			\draw[linedash] (C) -- (C2);
			\draw[linedash] (D) -- (D2);
			\draw[linedash] (E) -- (E2);
	
		\end{tikzpicture}
		\captionof{image}{Каноническое положение для правильной пятиугольной призмы}
		\label{fig:fig2}
	\end{figure}
\end{multicols}
	
\subsection{Вспомогательные функции}

\subsubsection{Функции для работы с координатами}

\function{verticesInGivenRange}{vertex, {startX, finishX, startY, finishY}}
Возвращает \texttt{true}, если двухмерная координата точки \texttt{vertex} вида \texttt{\{x,y\}} находится в некоторой прямоугольной области, иначе \texttt{false}.

\function{autoScale}{vertex3D, camera, vertex2D, {startX, finishX, startY, finishY, step, maxScale}}
Увеличивает свойство объекта \texttt{camera.scale} до тех пор, пока все двухмерные координаты \texttt{vertex2D} вида \texttt{\{x,y\}}  находится в некоторой прямоугольной области. \texttt{step} по умолчанию $0.1$.

\function{distanceFromPointToSegment}{point, segmentStart, segmentEnd}
Возвращает длину перпендикуляра между двухмерной точкой \texttt{point} вида \texttt{\{x,y\}} до прямой, проходящей через точки \texttt{segmentStart} и \texttt{segmentEnd}.

\subsubsection{Функции для работы с canvas}

\prototype[vertex, matrixConnections]{CanvasRenderingContext2D}{drawFigure}
Соединяет линиями точки массива \texttt{vertex} с элементами \texttt{\{x,y\}} в соответствии с матрицей связей \texttt{matrixConnections}, которая является массивом, содержащим в себе 0, 1 или массив step, указывающий на отрисовку пунктиром.

Пример матрицы связей:
\begin{lstlisting}[numbers=none]
	let matrixConnections = [
			[1],
			[strok, strok],
			[0, 0, strok],
			[1, 0, 0, 1],
			[0, 1, 0, 1, 1]
		];
	\end{lstlisting}

\prototype[\\vertex, matrixConnections]{CanvasRenderingContext2D}{drawFigureVer2}
Соединяет линиями точки массива \texttt{vertex} с элементами \texttt{\{x,y\}} в соответствии с матрицей связей \texttt{matrixConnections}. Эта матрица представляет собой объект, где каждое числовое поле соответствует номеру вершины в массиве \texttt{vertex}. В каждом поле находится массив номеров других вершин, с которыми должна быть соединена данная вершина.

Пример матрицы связей:
\begin{lstlisting}[numbers=none]
	let matrixConnections = {
		0: [1, [3, stroke], 5],
		2: [1, [3, stroke], 7],
		4: [[3, stroke], 5, 7],
		9: [1, 8, 10],
		11: [8, 10, 12],
		13: [5, 8, 12],
		15: [7, 10, 12],
	};
	\end{lstlisting}

\subsection{Этапы разработки шаблоны с вспомогательным чертежом по теме «Стереометрия»}

Для примера возьмём задание №27074~\cite{sdamgia}.

\includegraphics*[width= 0.8\linewidth]{2774}

Заготовка шаблона имеет вид.

\lstinputlisting[]{code/27074_1.js}

\begin{enumerate}
	\item Создадим объект класса \texttt{Parallelepiped} со случайной высотой, шириной и глубиной в заданном диапазоне. 
	\lstinputlisting[]{code/27074_2.js}
	\item Определим переменную \texttt{camera}, которая будет отвечать за положение наблюдателя. Спроецируем канонические координаты параллелепипеда на двухмерную плоскость при помощи функции \texttt{project3DTo2D}, отмасштабируем полученные координаты так, чтобы они занимали максимально заполняли иллюстрацию, функцией \texttt{autoScale}.
	\lstinputlisting[]{code/27074_3.js}
	\item Перемещаемся в середину иллюстрации. Отрисовываем фигуру функцией \texttt{drawFigure}, передав в неё матрицу связей для параллелепипеда. 
	\lstinputlisting[]{code/27074_4.js}
	\item Далее вырезаем из условия значения и заменяем их данными из класса. Впишем ответ. Обособляем имена фигур в \$\dots\$. Добавляем буквы на вершины параллелепипеда. Добавляем модификаторы          \texttt{NAtask.modifiers.assertSaneDecimals()} (исключает нецелый ответ) и
	\texttt{NAtask.modifiers.variativeABC(letter)} (заменяет все буквы в задании на случайные).
	\lstinputlisting[]{code/27074_5.js}
\end{enumerate}


\section*{Заключение}
\addcontentsline{toc}{section}{Заключение}
В данной работе были приведены архитектура проекта «Час ЕГЭ», его библиотеки
и примеры генерируемых задач. Обоснована релевантность проекта по сравнению
с другими открытыми ресурсами.



\begin{thebibliography}{5}
	\bibitem{chasdok1} Момот Е. А., Арахов Н. Д. Разработка и внедрение ПО для сбора статистики результатов подготовки к ЕГЭ по математике профильного уровня //Актуальные проблемы прикладной математики, информатики и механики. – 2021. – С. 1-2.
	\bibitem{egemath}Открытый банк задач ЕГЭ по Математике.Профильный уровень. – URL:  https://prof.mathege.ru/
	\bibitem{chasi}Пошаговая инструкция по созданию элементарных шаблонов. – URL:  https://math.vsu.ru/chas-ege/doc/shabl-b1-po-shagam.html
	\bibitem{fipi}Федеральный институт педагогических измерений. – URL:  https://fipi.ru/ege/otkrytyy-bank-zadaniy-ege
	\bibitem{ege} Единый государственный экзамен. – URL:  https://ru.wikipedia.org/wiki/Единый\_государственный\_экзамен
\end{thebibliography}

\section*{Приложение}

\addcontentsline{toc}{section}{Приложение}

\lstinputlisting[caption = 92.js, label={lst:92}, escapechar=|]{code/92.js}
\subsubsection*{Примеры генерируемых задач 92.js}

\task{Площадь параллелограмма $DLGS$ равна $45$. Точка $X$ – середина стороны $GS$. Найдите площадь трапеции $XGLD$.}{0,25}{7795388710467022n0}{92.js}
\task{Площадь параллелограмма $AHPJ$ равна $47$. Точка $G$ – середина стороны $PJ$. Найдите площадь трапеции $PGAH$.}{0,75}{7795388710467022n0}{92.js}
\task{Площадь параллелограмма $CHPN$ равна $16$. Точка $B$ – середина стороны $CN$. Найдите площадь трапеции $HPNB$.}{0,25}{006640344389165n0}{92.js}

\lstinputlisting[caption = 3011.js, label={lst:3011}, escapechar=|]{code/3011.js}
\subsubsection*{Примеры генерируемых задач 3011.js}   

\task{В правильной четырёхугольной пирамиде $OQSDX$ с основанием $QSDX$ боковое ребро равно $\sqrt{1489,5}$, сторона основания равна $39$. Найдите объём пирамиды.}{13689}{314379431351734n0}{3011.js}
\task{В правильной четырёхугольной пирамиде $VEBXI$ с основанием $EBXI$ боковое ребро равно $\sqrt{848,5}$, сторона основания равна $27$. Найдите объём пирамиды.}{5346}{438948036652505n0}{3011.js}
\task{В правильной четырёхугольной пирамиде $WYFDC$ с основанием $YFDC$ боковое ребро равно $\sqrt{1513}$, сторона основания равна $24$. Найдите объём пирамиды.}{6720}{087863431801837n0}{3011.js}

\lstinputlisting[caption = 2069.js, label={lst:2069}, escapechar=|]{code/2069.js}
\subsubsection*{Примеры генерируемых задач 2069.js}   

\task{Угол между биссектрисой и медианой прямоугольного треугольника, проведёнными из вершины прямого угла, равен $38^{\circ}$. Найдите больший угол прямоугольного треугольника. Ответ дайте в градусах.}{83}{975621288543568n0}{2069.js}
\task{Угол между биссектрисой и медианой прямоугольного треугольника, проведёнными из вершины прямого угла, равен $8^{\circ}$. Найдите меньший угол прямоугольного треугольника. Ответ дайте в градусах.}{37}{552764015073117n0}{2069.js}
\task{Угол между биссектрисой и медианой прямоугольного треугольника, проведёнными из вершины прямого угла, равен $31^{\circ}$. Найдите меньший угол прямоугольного треугольника. Ответ дайте в градусах.}{14}{673204479916529n0}{2069.js}

\lstinputlisting[caption = 29193.js, label={lst:27193}, escapechar=|]{code/27193.js}
\subsubsection*{Примеры генерируемых задач 27193.js}

\task{Найдите площадь поверхности многогранника, изображённого на рисунке (все двугранные углы – прямые).}{948}{020701664257409n0}{27193.js}
\task{Найдите объём многогранника, изображённого на рисунке (все двугранные углы – прямые).}{2871}{636805691610557n0}{27193.js}
\task{Найдите площадь поверхности многогранника, изображённого на рисунке (все двугранные углы – прямые).}{1454}{392642674576018n0}{27193.js}

\lstinputlisting[caption = 27764.js, label={lst:27764}, escapechar=|]{code/27764.js}
\subsubsection*{Примеры генерируемых задач 27764.js}   

\task{В треугольнике $RHC$ угол $R$ равен $67^{\circ}$, углы $H$ и $C$ – острые, биссектрисы $LH$ и $SC$ пересекаются в точке $X$. Найдите угол $CXH$. Ответ дайте в градусах.}{123,5}{7108522192090796n0}{27764.js}
\task{В треугольнике $YHA$ угол $Y$ равен $45^{\circ}$, углы $H$ и $A$ – острые, биссектрисы $HW$ и $AR$ пересекаются в точке $M$. Найдите угол $AMH$. Ответ дайте в градусах.}{112,5}{6232978139752687n0}{27764.js}
\task{В треугольнике $URF$ угол $U$ равен $7^{\circ}$, углы $R$ и $F$ – острые, биссектрисы $AR$ и $FQ$ пересекаются в точке $X$. Найдите угол $FXR$. Ответ дайте в градусах.}{93,5}{360610115591288n0}{27764.js}

\lstinputlisting[caption = 109.js, label={lst:109}, escapechar=|]{code/109.js}
\subsubsection*{Примеры генерируемых задач 109.js}

\task{В треугольнике $CBX$ $BP=5$, $CB=20$, $CX=BX$, $CP - высота$. Найдите косинус $BCX$.}{33,75}{8306742797956614n0}{109.js}
\task{В треугольнике $BNW$ $BW=NW$, $BY - высота$, $NY=6$, $BN=8$. Найдите косинус $NBW$.}{35,25}{4956012811026389n0}{109.js}
\task{Объём правильной четырёхугольной пирамиды $UOVSZ$ равен $25650$. Точка $B$ – середина ребра $UO$. Найдите объём треугольной пирамиды $BOVZ$.}{12}{1048766914435171n0}{109.js}


\end{document}

%Планиметрия
  %Вспомогательные функции
  %шаблоны
%Нейронные сети и шаблоны по теме «Стереометрия»
  %Современный прогресс генеративных текстовых нейросетей
  %Разработка библиотек с помощью Gpt-Chat
  %Применение ООП для разработки шаблонов
  %шаблоны

%Введение
%определение шаблона
%Основные сведения о проекте
  %1) используемые технологи
  %2)Внутренние библиотеки
  %3)Примеры шаблонов заданий(мои старые простые)
%Шаблоны на чтение графиков функции с произвольным параметром
  %вспомогательные функции добавленные к библиотеке
  %разработанные шаблоны
%Сплайны третьего порядка и шаблоны на чтение графика функции и её производной
  %понятие сплайна 3 порядка + про библиотеку
  %вспомогательные функцииЫ
  %шаблоны
%ВОЗМОЖНО ТОЛЬКО ЭТО

%7 делится на две главы. Задача с чертежом коорд.оси (предвор сведения, разработанные библ функции, шаблоны)
% вторая - задача на сравнение чисел (как устроено округление,autoLateX )
\subsection{Задачи №7 ОГЭ (координатная прямая)}

Наиболее значимая часть работы — это разработка функций для визуализации координатной прямой и точек на ней. 
Была создана универсальная функция \texttt{coordAxis\_drawMarkPoint}, позволяющая отображать засечки и подписи в разных режимах.
\section{Функция для отрисовки меток на координатной оси}

Одним из ключевых элементов реализации автоматической генерации заданий
стала функция \texttt{coordAxis\_drawMarkPoint}. Она предназначена для
отрисовки различных типов меток на координатной оси и их подписей.

\subsection{Назначение функции}
Функция решает задачу визуализации точек и вспомогательных обозначений на оси,
что является неотъемлемой частью заданий ОГЭ и ЕГЭ по математике.
С помощью данной функции возможно изображать:
\begin{itemize}
    \item закрашенные точки (``dot''),
    \item выколотые точки (``emptyDot''),
    \item засечки (``line''),
    \item отсутствие метки (``nothing'').
\end{itemize}

\subsection{Интерфейс функции}
Функция имеет следующий набор параметров:
\begin{itemize}
    \item \texttt{ct}~--- графический контекст Canvas,
    \item \texttt{coord}~--- координата по оси $X$,
    \item \texttt{text}~--- подпись для метки,
    \item \texttt{markForm}~--- форма метки: \texttt{dot}, \texttt{emptyDot}, \texttt{line}, \texttt{nothing},
    \item \texttt{textPosition}~--- расположение подписи: под осью (\texttt{underAxis}), над осью (\texttt{overAxis}), на оси (\texttt{onAxis}),
    \item \texttt{options}~--- дополнительные параметры (шрифт, цвет текста, толщина линии, смещение).
\end{itemize}

\subsection{Алгоритм работы}
\begin{enumerate}
    \item Сохраняются текущие параметры отрисовки (\texttt{fillStyle}, \texttt{strokeStyle}, \texttt{font}, \texttt{lineWidth}).
    \item Устанавливаются новые параметры, переданные в \texttt{options}.
    \item В зависимости от параметра \texttt{markForm} рисуется выбранный элемент:
    \begin{itemize}
        \item точка~--- закрашенный круг,
        \item выколотая точка~--- окружность с заливкой белым цветом внутри,
        \item засечка~--- вертикальная черта,
        \item отсутствие~--- элемент не отрисовывается.
    \end{itemize}
    \item В зависимости от параметра \texttt{textPosition} подпись размещается под осью, над осью или на линии оси.
    \item Восстанавливаются исходные параметры графического контекста.
\end{enumerate}

\subsection{Взаимодействие с другими функциями}
Данная функция является частью связки:
\begin{itemize}
    \item \texttt{coordAxis\_prepare}~--- подготавливает область для оси и рисует стрелку,
    \item \texttt{coordAxis\_drawAuto}~--- автоматически вычисляет масштаб оси и вызывает \texttt{coordAxis\_drawMarkPoint} для всех точек.
\end{itemize}

\subsection{Особенности реализации}
\begin{itemize}
    \item Поддержка как закрашенных, так и выколотых точек позволяет формировать задания с открытыми и закрытыми интервалами.
    \item Восстановление исходных параметров гарантирует корректную работу при множественной отрисовке.
    \item Возможность смещения текста по оси $X$ помогает избежать наложений подписей.
\end{itemize}

\subsection{Пример использования}
\begin{verbatim}
// Отрисовка закрашенной точки A с подписью под осью
coordAxis_drawMarkPoint(ct, 100, "A", "dot", "underAxis");

// Отрисовка выколотой точки B с подписью над осью
coordAxis_drawMarkPoint(ct, 200, "B", "emptyDot", "overAxis");
\end{verbatim}

%%Программная реализация (на языке Javascript) алгоритмов
%% генерации фонда оценочных средств по математике
\section{Планиметрия}
В это главе мы приводим вспомогательные функции и алгоритм написания шаблона по планиметрии
\subsection{Вспомогательные функции}
\subsubsection{Функции для работы с массивами}
\prototype[repeat]{Array}{permuteCyclic}
Возвращает массив после циклической перестановки.
В листинге \ref{lst:105} в строке \ref{line:permuteCyclic} функция используется для перестановки букв в названии угла.
%TODO: добавить пример про многоугольник (именование)

\begin{lstlisting}
    let array = [1, 2, 3, 4, 5];

    array.permuteCyclic(1);
    // [5, 1, 2, 3, 4]

    array.permuteCyclic(-2);
    // [3, 4, 5, 1, 2]

    array.permuteCyclic(0);
    // [1, 2, 3, 4, 5]
    \end{lstlisting}

\prototype{Array}{mt\_coordinatesOfIntersectionOfTwoSegments\\}
Возвращает координаты пересечения двух отрезков, задаваемых первыми парами точек из массива.
В листинге \ref{lst:27193} в строке \ref{line:mt_coordinatesOfIntersectionOfTwoSegments} функция используется для нахождения точки пересечения проекций рёбер составного многогранника.
%А причём тут это. Заменить на планиметрию

\begin{lstlisting}
    let array = [{x:0,y:5},{x:-4,y:4},{x:1,y:10},{x:-3,y:6}];

    array.mt_coordinatesOfIntersectionOfTwoSegments()
    //{ x: -5.333333333333333, y: 3.666666666666667, status: false }
    //Отрезки не пересекаются, но прямые проходящие через них пересекаются в точке {x,y}

    array = [{x:0,y:5},{x:-4,y:4},{x:1,y:1},{x:-3,y:6}];
    array.mt_coordinatesOfIntersectionOfTwoSegments()
    //{ x: -1.8333333333333333, y: 4.541666666666667, status: true }
    //Отрезки пересекаются в точке {x,y}
        
\end{lstlisting}

\prototype[separator]{Array}{shuffleJoin}
Перемешивает и соединяет массив с разделителем \texttt{separator}. \texttt{separator} по умолчанию пустая строка. Функция используется в \ref{lst:3011}
в строке \ref{line:shuffleJoin} для отображения условий задачи в случайном порядке.

\begin{lstlisting}
    let array = ['A', 'B', 'C', 'D',];
    array.shuffleJoin();
    //ADBC

    array.shuffleJoin('; ');
    //C; D; B; A 
\end{lstlisting}

\prototype[separator]{Array}{joinWithConjunction}
Соединяет массив запятыми и соединяет два последних элемента союзом «и».

\begin{lstlisting}
    let array = ['A', 'B', 'C', 'D',];

    array.joinWithConjunction();
    //A, B, C и D
\end{lstlisting}

\subsubsection{Функции для работы с числами}
\prototype{Number}{perfectCubicMultiplier}
Возвращает максимальный делитель данного числа, куб которого также является делителем данного числа.

\begin{lstlisting}
    let number = 81;

    number.perfectCubicMultiplier()
    //3

    number = 36;
    number.perfectCubicMultiplier()
    //1

    number = -27;
    number.perfectCubicMultiplier()
    //3
\end{lstlisting}

\prototype[p1, p2]{Number}{texcbrt}
TeX-представление кубического корня из данного числа.\\
Если данное число - полный куб, то корень из числа.\\
Если \texttt{p1}, то из-под корня будут вынесены возможные множители.\\
Если \texttt{p1}, \texttt{p2} и из-под корня выносится единица, то она будет опущена\\
%TODO:Приложить картинку

\subsubsection{Функции для работы с canvas}
\prototype[vertex,\\ fillStyle]{CanvasRenderingContext2D}{drawSection}
Заполняет область цветом \texttt{fillStyle} по вершинам из массива \texttt{vertex}.

\begin{lstlisting}
    let paint1 = function(ctx) {
        let h = 400;
        let w = 400;
        ctx.drawCoordinatePlane(w, h, {
            hor: 1,
            ver: 1
        }, {
            x1: '1',
            y1: '1',
            sh1: 16,
        },30);
        ctx.scale(30, -30);
        ctx.drawSection([[1, 3],[-3, 0],[-2, -2],[1, -1],[5, 1],[4, 4],[3, 2]]);

        ctx.drawSection([[-2, 0],[-1, 1],[-4, 3],[-1, 5],[1, 1],[5, 2],[4, -6],[0, 0],[-4, -2],]);
    };
    \end{lstlisting}
    \includegraphics[width=0.4\textwidth]{drawSection-1.png}
    \includegraphics[width=0.4\textwidth]{drawSection-2.png}

\prototype[x, y,\\ angle, length]{CanvasRenderingContext2D}{drawLineAtAngle}
Рисует отрезок длины \texttt{length} под углом angle ( в радианах). Пример использования можно найти в листинге \ref{lst:2069} в строках \ref{line:drawLineAtAngle-1} и \ref{line:drawLineAtAngle-2} (применяется для отрисовки биссектрисы). 

\prototype[x1, y1, x2, y2, length, quantity]{CanvasRenderingContext2D}{strokeInMiddleOfSegment\\}
Ставит штрихи длины \texttt{length} на середине отрезка перпендикулярно ему. Функция используется в листинге \ref{lst:2069} в строках \ref{line:strokeInMiddleOfSegment-1}-\ref{line:strokeInMiddleOfSegment-2} для обозначения равных по длине сторон треугольника.

\prototype[x, y, angle, letter, length, maxLength]{CanvasRenderingContext2D}{markSegmentWithLetter\\}
Вспомогательная функция для отрисовки текста около некоторого отрезка.

\prototype[x1, y1, x2, y2, letter, length, maxLength]{CanvasRenderingContext2D}{signSegmentInMiddle\\}
Рисует строку \texttt{letter} на середине отрезка. В листинге \ref{lst:27193} в строках \ref{line:signSegmentInMiddle-1} - \ref{line:signSegmentInMiddle-2} функция используется для отображения длин рёбер многогранника.

\prototype[coordinates, radius]{CanvasRenderingContext2D}{arcBetweenSegments\\}
Рисует знак угла между двумя отрезками в месте их пересечения. \texttt{coordinates} - массив вида \texttt{[x1, y1, x2, y2]}.

\begin{lstlisting}
    let paint1 = function(ctx) {
        let h = 400;
        let w = 400;
        ctx.drawCoordinatePlane(w, h, {
            hor: 1,
            ver: 1
        }, {
            x1: '1',
            y1: '1',
            sh1: 16,
        }, 30);
        ctx.scale(30, -30);

        ctx.lineWidth = 2 / 30;
        ctx.drawLine(1, 5, 3, -2);
        ctx.drawLine(3, -2, 5, -3);
        ctx.arcBetweenSegments([1, 5, 3, -2, 5, -3], 2);

        ctx.drawLine(2, -5, -4, -2);
        ctx.drawLine(1, -2, -3, -6);
        ctx.arcBetweenSegments([2, -5, -4, -2,  -3, -6,1, -2,], 1);

        ctx.drawLine(1, 5, 1, -2);
		ctx.drawLine(1, -2, 5, -2);
		ctx.strokeStyle = om.secondaryBrandColors.iz();
		ctx.arcBetweenSegments([1, 5, 1, -2, 5, -2], 3);

    };
\end{lstlisting}

\includegraphics[width=0.4\textwidth]{arcBetweenSegments-1.png}    
\includegraphics[width=0.4\textwidth]{arcBetweenSegments-2.png}    

\prototype[coordinates, radius, number, step]{CanvasRenderingContext2D}{arcBetweenSegmentsCount\\}
Рисует знак угла между двумя отрезками в месте их пересечения \texttt{number} раз с отступом \texttt{step}. В листинге \ref{lst:27764} в строках \ref{line:arcBetweenSegmentsCount-1} - \ref{line:arcBetweenSegmentsCount-2} используется для обозначения двух равных углов.

\prototype[x, y, radiusX, radiusY, rotation, startAngle, endAngle,\\ anticlockwise]{CanvasRenderingContext2D}{drawEllipse\\}
Рисует эллипс.

\prototype[x, y, radius, startAngle, endAngle, anticlockwise]{CanvasRenderingContext2D}{drawArc\\}
Рисует дугу.

\subsubsection{Элементы декларативного программирования}

%TODO: определение, по сути трясёт случайные числа, пока не подойдёт. Ошибки видны только для программитса при отладке.

Во время разработки шаблонов по теме «Графики функции» требовалось много раз переопределять коэффициенты функций через циклы 
\texttt{while} или \texttt{do\dots while}, пока они не начнут соответствовать заданным условиям (видимость графика, сливание его с осями, видимость целых точек). Это часто приводило к бесконечной работе шаблона, при этом сложно было определить, какое условие не выполняется.

Для этого было разработано окружение \texttt{retryWhileUndefined} для шаблонов, которое бы перезапускало их не более \texttt{maxIterations} раз, если одно из условий не удалетворено. 

\function{retryWhileUndefined}{theFunction, maxIterations}

Но всё равно было тяжело определить, почему шаблон перезапускается. Для этого было разработано более совершенное окружение \texttt{retryWhileError}, которое не только могло бы ограничивать количество перезапусков, но и фиксировать, какие проверки не были пройдены и выводить их на экран.

\function{retryWhileError}{theFunction, maxIterations,maxCollectedErrors}

Для окружения были написаны функции-утверждения, которые имеют структуру : условие не выполнено - записать ошибку - перезапустить шаблон. Если максимальное количество повторений достигнуто, то вывести накопившиеся ошибки и количество их появлений. 

\function{genAssert}{condition, message}
    Если условие \texttt{condition} ложно, то шаблон перезапускается. 

\function{genAssertNonempty}{array, message}
    Если массив \texttt{array} пуст, то шаблон перезапускается.

\function{genAssertZ1000}{number, message}
    Если число \texttt{number} имеет более 3 знаков после запятой, то шаблон перезапускается.
    
\function{genAssertIrreducible}{numerator, denominator, message}
    Если дробь \texttt{numerator/denominator} сократима, то шаблон перезапускается.

\function{genAssertSaneDecomposition}{number, maxFactor, message}
    Если \texttt{number} число не раскладывается на простые множители, не более одного из которых превосходит \texttt{maxFactor}, то шаблон перезапускается. 

\subsection{Этапы разработки шаблона со вспомогательным чертежом по теме «Планиметрия»}

Для примера возьмём задание №19416~\cite{egemath}.
\\
\textbf{Задача №19376.}
В треугольнике $ABC$ известно, что ${AC=BC}$, $AB=16$, $AH$ --- высота, $BH=4$. Найдите косинус угла $BAC$.\\ 

Заготовка шаблона имеет вид.

\lstinputlisting[]{code/109_1.js}

\begin{enumerate}
    \item Начнём с отрисовки чертежа для задания. Отметим стороны треугольника так, чтобы он лежал в центре холста, а до краёв оставалось 10-20px. При отрисовке используем функцию \texttt{drawLine}. Добавим высоту 
    \lstinputlisting[]{code/109_2.js} 
    \item Добавим на рисунок штрихи, указывающие на равенство сторон и обозначение прямого угла при помощи функций strokeInMiddleOfSegment и arcBetweenSegments соответственно. И подпишем вершины и точку перпендикуляра. Добавим модификатор \texttt{NAtask.modifiers.variativeABC(vertices)}, который заменяет все буквы в задании на случайные.
    %TODO: пояснить за параметр

    \lstinputlisting[]{code/109_3.js} 
    \item Теперь добавим ответ в задание. Проверим при помощи \texttt{genAssertZ1000}, что ответ имеет не более трёх знаков после запятой (если иначе шаблон запускается заново). Поместим все буквы и числа в \$\dots\$. Все условия из задачи преобразуем в массив и соединим случайным образом с помощью функции \texttt{shuffleJoin}.
    \lstinputlisting[]{code/109_4.js} 
\end{enumerate}


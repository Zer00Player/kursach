
\section*{Введение}
\addcontentsline{toc}{section}{Введение}
Единый государственный экзамен (ЕГЭ)~— централизованно проводимый в Российской
Федерации экзамен в средних учебных заведениях — школах, лицеях и гимназиях,
форма проведения ГИА (Государственной Итоговой Аттестации) по образовательным программам среднего общего образования.
Служит одновременно выпускным экзаменом из школы и вступительным экзаменом в вузы.
%Черновое вступление. Переписать позже полностью своими словами
Но за время обучения в 9, 10 и 11 классе при подготовке к ОГЭ и ЕГЭ школьники сталкиваются с дефицитом заданий по определённым категориям.
Так, за последние 5 лет в список заданий ЕГЭ были добавлены новые задания под номером 1 по теме «Округление с недостатком» и «Округление с избытком», так же задания под номером 15 «проценты и округление»,21 задания « Текстовые задачи», количество которых для прорешивания было очень мало. 
Ко всему прочему в задании номер 7 по теме «Числовые неравенства, координатная прямая -числа на прямой» банк заданий расходуется при подготовке с невероятной скоростью:
так как это преимущественно графические задания, решение их занимает менее минуты, а их составление вручную занимает несоразмерно много времени. ОГЭ и ЕГЭ является относительно неизменяемым экзаменом, поэтому все материалы, которые уже были выложены в открытый доступ, имеют полные решения, что приводит к списыванию учениками.

При этом существуют задания со вспомогательным чертежом. Чаще всего для целого ряда заданий используется одна и та же иллюстрация, которая не всегда соответствуют условиям задачи, а иногда отвлекает от решения.
Проект «Час ЕГЭ» позволяет решить все эти проблемы.

«Час ЕГЭ» — компьютерный образовательный проект, разрабатываемый при математическом
факультете ВГУ в рамках «OpenSource кластера» и предназначенный для помощи учащимся
старших классов при подготовке к тестовой части единого государственного экзамена.
%%ссылочки на доклады
Задания в «Час ЕГЭ» генерируются случайным образом по специализированным алгоритмам,
называемых шаблонами, каждый из которых
охватывает множество вариантов соответствующей ему задачи. Для
пользователей
предназначены четыре оболочки (режима работы): «Случайное задание», «Тесты на печать»,
«Полный тест» и «Мини-интеграция».
«Час ЕГЭ» является полностью открытым (код находится под лицензией GNU GPL 3.0)
и бесплатным.
В настоящее время в проекте полностью реализованы тесты по математике с кратким
ответом (бывшая «часть В»).~\cite{fipi}
Планируется с течением времени включить в проект тесты по другим предметам школьной
программы.

%Первая глава этой работы посвящена обзору вспомогательных функций, которые ускоряют написание шаблонов по теме «Планиметрия» и введению в проект элементов декларативного программирования. Также приведён алгоритм написания шаблона с чертежом.

Первую главу посвятим обзору шаблонов для номеров 21,15, 1 из ЕГЭ базовой.
Во второй главе рассмотрим функцию добавленному для упрощения отрисовки прямых для 7 задания ОГЭ.

%Вторая глава представляет решение проблемы отрисовки фигур в трёхмерном пространстве на языке программирования JavaScript; рассказывает о применении объектно-ориентированного программирования для упрощения написания шаблонов с чертежом; затрагивает вопрос об написании программного кода при помощи нейросетей; приводит обзор вспомогательных функций и алгоритм написания шаблона по теме «Стереометрия». 

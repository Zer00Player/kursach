\section{Генерация текстовых задач}

В Первом разделе представлены работы, связанные Текстовыми задачами. 
%Работа велась на языке \texttt{JavaScript}, с использованием вспомогательных функций для визуализации и генерации условий.
Так как в 
%15, 21 "текстовые задачи"
%рассказать про декорации, про chislitlx(пулл реквест с r2), сочетания 
%предварительные сведения (функции, кратко ,генассерт и тп)
%платформа час-ЕГЭ (то что сделала сама "хвалюсь") для "реализации задачи мною были разработаны" и тп, 
% то что существует - описываю как факт (а.к.а предварительные сведения)
% 4 1)отдельно (формулы) естественно появляется обратная задача. 2)Сослаться на Латех (MathJax) основные команды которыми пользуемся в Латехе texfrag и тп 
%7 делится на две главы. Задача с чертежом коорд.оси (предвор сведения, разработанные библ функции, шаблоны)
% вторая - задача на сравнение чисел (как устроено округление,autoLateX )
% в заключении написать количество вып задач по категориям 
%в данной главе (тото это тамто и тп)
% не менее 10 шаблонов сумарно затронув каждую категорию 
% где-то 45-50 страниц....
% про расвнение чисел очень долго целых чисел не было, все числа были дробные, 0.1+0.2!=0.3 надо быть осторожным и тп расскажи!! 
%захожу в калатог, через меню переключаемся на тест на печать внимательно (?) выставить нулями все задания кроме того что нужно ставить галочку " экспорт в латех" 
%ставим запуск,жду,думаю, падает зипка с задачами в латехе, в перемешку 
%исходная задача/код/полученная задача (сгенерин текст)
При построении текстовых шаблонов задач важно обеспечить как вариативность формулировок, так и корректность числовых и языковых выражений. Для этого в проекте применяются следующие инструменты.

\subsection{Декорации}

Чтобы увеличить количество уникальных вариантов задач, сохраняя их математическую суть, используются так называемые ``декорации''~--- элементы окружения, которые можно менять без потери смысла задачи: имена персонажей, место действия, цель, контекст. 

Для этого создаются массивы строк, а функция \texttt{iz()} случайным образом выбирает элемент из массива.  
Пример:
\begin{lstlisting}
let contract = ['о дружбе', 'во избежание двойного налогообложения',
    'о безвизовом режиме', 'об экологической среде',
    'по гуманитарным вопросам', 'по вопросам безопасности'].iz();
\end{lstlisting}
\textbf{Задача №514913.}

Если необходимо выбрать два случайных, но не повторяющихся элемента, используется форма \texttt{iz(2)}:  
\begin{lstlisting}
let tapeName = sklonlxkand(['лента', 'верёвка', 'нитка'].iz(2));
\end{lstlisting}
\textbf{Задача №2434.}

\subsection{Склонение существительных}

В силу особенностей русского языка слова должны корректно изменяться по падежам и числам. Для этого применяется функция \texttt{sklonlxkand}, позволяющая генерировать правильные словоформы.  

После выбора слова к нему можно обратиться по падежу и числу. Например, \texttt{.ie} означает именительный падеж, единственное число.  

Пример использования:
\begin{lstlisting}
let typeOfFlowerInVases =
  sklonlxkand(['роза','гвоздника','ромашка','лилия',
               'мак','ирис','лаванда','мимоза'].iz());

NAtask.setTask({
  text: 'На прилавке цветочного магазина стоят три вазы с '
    + typeOfFlowerInVases.tm + ': ' + vaseColor[0] + 'ая, '
    + vaseColor[1] + 'ая, ' + vaseColor[2] + 'ая.' 
    + ' Слева от ' + vaseColor[2] + 'ой вазы '
    + chislitlx(leftOfThirdVase, typeOfFlowerInVases, '$')
    + ', справа от ' + vaseColor[0] + 'ой вазы '
    + chislitlx(righttOfFirstVase, typeOfFlowerInVases, '$') + '. '
    + 'Всего в вазах '
    + chislitlx(allFlowerInVases, typeOfFlowerInVases, '$') + '. '
    + 'Сколько ' + typeOfFlowerInVases.rm + ' в '
    + vaseColor[1] + 'ой вазе?',
  answers: secondVaseCountFlower,
});
\end{lstlisting}
\textbf{Задача №515842.}

\subsection{Связка числа и существительного}

Для корректного согласования числительных с существительными используется функция \texttt{chislitlx}. Она автоматически выбирает нужную форму слова в зависимости от числа.  
Пример:
\begin{lstlisting}
let numberOfApartamentPerFloor = sluchch(3, 12, 1);
NAtask.setTask({
  text: 'В доме, в котором живет ' + nameOfPerson + ', ' +
    '$' + floorNumber + '$ этажей и несколько подъездов. На каждом этаже находится по ' +
    chislitlx(numberOfApartamentPerFloor, 'квартира', '$') + '. '
    + nameOfPerson + ' живет в квартире №' + '$' + apartamentNumber + '$' + '. '
    + 'В каком подъезде живет ' + nameOfPerson + '? ',
  answers: '$' + (apartamentNumber /
             (floorNumber * numberOfApartamentPerFloor)).ceil() + '$',
});
\end{lstlisting}
\textbf{Задача №77351.}

\subsection{Проверка условий (генерация утверждений)}
Для контроля корректности задачи применяются функции \texttt{genAssert} и её вариации. Принцип работы:  
\begin{itemize}
    \item если условие не выполнено~--- фиксируется ошибка;
    \item шаблон перезапускается;
    \item если достигнуто максимальное количество перезапусков, выводятся накопившиеся ошибки с указанием количества.
\end{itemize}

Пример использования:
\begin{lstlisting}
let students = sl(200, 10000, 10);
let percent = sl(10, 90, 1); 
genAssert(students.kratno(100 / percent),
          "количество учащихся не кратно 100/процент");
\end{lstlisting}
\textbf{Задача №77340.}

Также часто используется функция \function{genAssertZ1000}{number, message}, проверяющая точность чисел: если у числа более трёх знаков после запятой, шаблон перезапускается.  
Пример:
\begin{lstlisting}
let prise = sl(100, 2000, 10);
let percentSecondMonth = sl(10, 90, 1);
let percentThirdMonth = sl(10, 90, 1);

let middlePrise = prise * (1 + [1, -1][randFirst] * 0.01 * percentSecondMonth);
let finalePrise = middlePrise * (1 + [1, -1][randSecond] * 0.01 * percentThirdMonth);

genAssertZ1000(finalePrise / 10,
               'Число имеет более 2 знаков после запятой');
\end{lstlisting}
\textbf{Задача №77349.}
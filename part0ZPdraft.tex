\section{Генерация задач для ЕГЭ и ОГЭ}

В этом разделе представлены пару работ, связанные с автоматизацией составления заданий по математике. 
Работа велась на языке \texttt{JavaScript}, с использованием авторских вспомогательных функций для визуализации и генерации условий.

\subsection{Задачи №21 ЕГЭ (задачи на смекалку)}

\begin{lstlisting}[language=JavaScript]
(function () { 
    'use strict'; retryWhileError(function ()
    { NAinfo.requireApiVersion(0, 2);

    let firstCountries = sl(3, 9, 1);
    let secondCountries = sl(3, 9, 1);

    let otherCountriesFirst = firstCountries - 1;
    let otherCountriesSecond = secondCountries - 1;

    let numberWords = ['три', 'четыре', 'пять', 'шесть', 'семь', 'восемь', 'девять'];
    let previousNumberWords = ['две', 'три', 'четыре', 'пять', 'шесть', 'семь', 'восемь'];

    let firstCountriesWord = lx[numberWords[firstCountries - 3]].i; 
    let secondCountriesWord = lx[numberWords[secondCountries - 3]].r;

    let firstOtherCountriesWord = lx[previousNumberWords[firstCountries - 3]].t; 
    let secondOtherCountriesWord = lx[previousNumberWords[secondCountries - 3]].t; 

    let contract = ['о дружбе', 'во избежание двойного налогообложения', 'о безвизовом режиме',
     'об экологической среде', 'по гуманитарным вопросам', 'по вопросам безопасности'].iz();
    let result = firstCountries * otherCountriesSecond + secondCountries * otherCountriesFirst; 

    genAssert(result.kratno(2), "количество подписей должно быть кратно двум"); 

    NAtask.setTask({ t
    ext: 'Из ' + '$' + (firstCountries + secondCountries) + '$' + ' стран ' + firstCountriesWord + ' подписали договор ' +
     contract + ' ровно с ' + secondOtherCountriesWord + ' другими странами, ' + 'а каждая из оставшихся ' + secondCountriesWord +
      ' — ровно с ' + firstOtherCountriesWord + '. ' + 'Сколько всего было подписано договоров?', 
      answers: result / 2, 
      }); 
      }); 
      })();
\end{lstlisting}

\subsection{Задачи №15 ЕГЭ (проценты и дроби)}

Данный проект посвящён автоматической генерации задач по процентам и дробям. 
В качестве основы использовались реальные прототипы заданий из банка ФИПИ.

\begin{lstlisting}[language=JavaScript]
(function () { 
    'use strict'; 
    retryWhileError(function () 
    { NAinfo.requireApiVersion(0, 2); 
    
    let putOnAccount = sl(500, 10000, 100); 
    let persent = sl(7, 30, 1); 

    let bank = ['Сберегательный банк', 'Банк'].iz(); 
    let bankName = ['Бритва Оккама', 'Угол перспективы', 'Паутина', 'Динамика роста', 'Вожжи коммерции', ' Сыграй в ящик', 'Копилка', 
    'Золотое сечение', 'Профит', 'Один процент разности', 'Пол-царства', 'Падает вверх', 'Из грязи В князи'].iz(); 
    let accountEndOfYears = putOnAccount + putOnAccount * (persent * 0.01); 
    
    NAtask.setTask({ text: bank + ' "' + bankName + '" начисляет на срочный вклад ', 
    questions: [ 
        { text: '$' + persent + '$' + '% годовых. ' + 'Вкладчик положил на счет ' + '$' + putOnAccount + '$' + ' р. ' +
         'Какая сумма будет на этом счёте через год, ' + 'если никаких операций со счётом проводиться не будет', 
         answers: accountEndOfYears, 
        }, 
        { text: '$' + persent + '$' + '% годовых. ' + 'Вкладчик положил на счёт некоторую сумму. ' +
         'Со счётом не проводилось никаких операций и через год на нём лежала сумма равная ' + '$' + accountEndOfYears + '$' + ' р. ' + 
         'Сколько рублей изначально положил на счёт вкладчик', 
         answers: putOnAccount, }, 
        { text: 'проценты годовых. ' + 'Вкладчик положил на счет ' + '$' + putOnAccount + '$' + ' р. ' + 
        'Со счётом не проводилось никаких операций и через год на нём лежала сумма равная ' + '$' + accountEndOfYears + '$' + ' р. ' + 
        'Сколько процентов годовых начислил банк на этот вклад', 
        answers: persent, 
        }, 
        ], 
        postquestion: '?', 
        }); 
        }, 100); 
        })();
\end{lstlisting}

\subsection{Задачи №4 ЕГЭ (преобразование выражений)}

\begin{lstlisting}[language=JavaScript]
(function () { 
    'use strict'; 
    retryWhileError(function () 
    { NAinfo.requireApiVersion(0, 2); 
    let key = '506276'; 
    let preference = ['geometric', 'quadratic', 'harmonic']; 
    let rand = getSelectedPreferenceFromList(key, preference);
    genAssert(!sl(0, 25) || rand !== 2, 'Среднее гармоническое появляется слишком часто'); 
     
    let the_orderToFind = decor.orderToFind.iz(); 
    let word = ['геометрическое', 'квадратичное', 'гармоническое'][rand]; 
     
    let a = sl(1, 50); let b = slKrome([a], 1, 50); let c = slKrome([a, b], 1, 50); 
    let answer = [(a * b * c).cbrt(), ((a ** 2 + b ** 2 + c ** 2) / 3).sqrt(), 3 / (a + b + c)][rand];
     
    genAssertZ1000(answer, 'Должно быть не более 3 знаков после запятой'); 
     
    NAtask.setTask({ text: 'Среднее ' + word + ' трёх чисел $a$, $b$ и $c$ вычисляется по формуле $ ' +
      ['g = \\sqrt[3]{abc}', 'q = \\sqrt{\\frac{a^2+b^2+c^2}{3}}', 'h = \\left(\\frac{\\frac{1}{a}+\\frac{1}{b}+\\frac{1}{c}}{3} \\right)^{-1}'][rand] + 
      '$. ' + the_orderToFind.toZagl() + ' среднее ' + word + ' чисел $' + [a, a, '\\frac{1}{' + a + '}'][rand] + '$, $' + [b, b, '\\frac{1}{' + b + '}'][rand] + 
      '$, $' + [c, c, '\\frac{1}{' + c + '}'][rand] + '$.',
       answers: answer, 
       preference: preference, 
       }); 
       NAtask.modifiers.allDecimalsToStandard();
    }, 2000); 
    })();
\end{lstlisting}

\subsection{Задачи №7 ОГЭ (координатная прямая)}

Наиболее значимая часть работы — это разработка функций для визуализации координатной прямой и точек на ней. 
Была создана универсальная функция \texttt{coordAxis\_drawMarkPoint}, позволяющая отображать засечки и подписи в разных режимах.
\section{Функция для отрисовки меток на координатной оси}

Одним из ключевых элементов реализации автоматической генерации заданий
стала функция \texttt{coordAxis\_drawMarkPoint}. Она предназначена для
отрисовки различных типов меток на координатной оси и их подписей.

\subsection{Назначение функции}
Функция решает задачу визуализации точек и вспомогательных обозначений на оси,
что является неотъемлемой частью заданий ОГЭ и ЕГЭ по математике.
С помощью данной функции возможно изображать:
\begin{itemize}
    \item закрашенные точки (``dot''),
    \item выколотые точки (``emptyDot''),
    \item засечки (``line''),
    \item отсутствие метки (``nothing'').
\end{itemize}

\subsection{Интерфейс функции}
Функция имеет следующий набор параметров:
\begin{itemize}
    \item \texttt{ct}~--- графический контекст Canvas,
    \item \texttt{coord}~--- координата по оси $X$,
    \item \texttt{text}~--- подпись для метки,
    \item \texttt{markForm}~--- форма метки: \texttt{dot}, \texttt{emptyDot}, \texttt{line}, \texttt{nothing},
    \item \texttt{textPosition}~--- расположение подписи: под осью (\texttt{underAxis}), над осью (\texttt{overAxis}), на оси (\texttt{onAxis}),
    \item \texttt{options}~--- дополнительные параметры (шрифт, цвет текста, толщина линии, смещение).
\end{itemize}

\subsection{Алгоритм работы}
\begin{enumerate}
    \item Сохраняются текущие параметры отрисовки (\texttt{fillStyle}, \texttt{strokeStyle}, \texttt{font}, \texttt{lineWidth}).
    \item Устанавливаются новые параметры, переданные в \texttt{options}.
    \item В зависимости от параметра \texttt{markForm} рисуется выбранный элемент:
    \begin{itemize}
        \item точка~--- закрашенный круг,
        \item выколотая точка~--- окружность с заливкой белым цветом внутри,
        \item засечка~--- вертикальная черта,
        \item отсутствие~--- элемент не отрисовывается.
    \end{itemize}
    \item В зависимости от параметра \texttt{textPosition} подпись размещается под осью, над осью или на линии оси.
    \item Восстанавливаются исходные параметры графического контекста.
\end{enumerate}

\subsection{Взаимодействие с другими функциями}
Данная функция является частью связки:
\begin{itemize}
    \item \texttt{coordAxis\_prepare}~--- подготавливает область для оси и рисует стрелку,
    \item \texttt{coordAxis\_drawAuto}~--- автоматически вычисляет масштаб оси и вызывает \texttt{coordAxis\_drawMarkPoint} для всех точек.
\end{itemize}

\subsection{Особенности реализации}
\begin{itemize}
    \item Поддержка как закрашенных, так и выколотых точек позволяет формировать задания с открытыми и закрытыми интервалами.
    \item Восстановление исходных параметров гарантирует корректную работу при множественной отрисовке.
    \item Возможность смещения текста по оси $X$ помогает избежать наложений подписей.
\end{itemize}

\subsection{Пример использования}
\begin{verbatim}
// Отрисовка закрашенной точки A с подписью под осью
coordAxis_drawMarkPoint(ct, 100, "A", "dot", "underAxis");

// Отрисовка выколотой точки B с подписью над осью
coordAxis_drawMarkPoint(ct, 200, "B", "emptyDot", "overAxis");
\end{verbatim}